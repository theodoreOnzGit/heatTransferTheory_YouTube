\documentclass[11pt]{article}

%%%% for pictures %%%%
\usepackage{graphicx}
\usepackage{float}

%%% for the degree sign%%%
\usepackage{siunitx}

%%%% for matrices and vectors%%%%

\usepackage{amsmath}

%%%% for citations %%%%%%
\usepackage[backend=biber,
style=apa,
sorting=none,
isbn=true,
doi=true,
url=true,
]{biblatex}\addbibresource{bibliography.bib}

%%%%% for citep

%%%% https://www.overleaf.com/learn/latex/natbib_citation_styles



\begin{document}

\title{Convection BL YouTube}
\author{Theodore Ong}
\date{\today}
\maketitle


\part{Some Links before we start}

\begin{verbatim}
https://en.wikipedia.org/wiki/Navier%E2%80%93Stokes_equations

Matrices in LaTeX
https://www.overleaf.com/learn/latex/Matrices

Tensors in LaTeX

Navier Stokes Equations
https://www.comsol.com/multiphysics/navier-stokes-equations
https://en.wikipedia.org/wiki/Navier%E2%80%93Stokes_equations

Github
https://github.com/theodoreOnzGit/heatTransferTheory_YouTube
\end{verbatim}



\part{Hydrodynamics}
\section{Navier Stokes Equations}
$$\frac{\partial }{\partial t} u + u \frac{\partial}{\partial x} u + v \frac{\partial}{\partial y} u + w \frac{\partial}{\partial z} u - \nu ( \frac{\partial^2}{\partial x^2} u + \frac{\partial^2}{\partial y^2} \ u + \frac{\partial^2}{\partial z^2} u) = - \frac{1}{\rho_0} \frac{\partial P}{\partial x} +g_x$$

$$\frac{\partial }{\partial t} v + u \frac{\partial}{\partial x} v + v \frac{\partial}{\partial y} v + w \frac{\partial}{\partial z} v - \nu ( \frac{\partial^2}{\partial x^2} v + \frac{\partial^2}{\partial y^2} \ v + \frac{\partial^2}{\partial z^2} v) = - \frac{1}{\rho_0} \frac{\partial P}{\partial y} +g_y$$

$$\frac{\partial }{\partial t} w + u \frac{\partial}{\partial x} w + v \frac{\partial}{\partial y} w + w \frac{\partial}{\partial z} w - \nu ( \frac{\partial^2}{\partial x^2} w + \frac{\partial^2}{\partial y^2} \ w + \frac{\partial^2}{\partial z^2} w) = - \frac{1}{\rho_0} \frac{\partial P}{\partial z} +g_z$$

\section{Boundary Layer Equations (Laminar)}
$$0 = - \frac{\partial P}{\partial y} $$

$$\frac{\partial }{\partial t} u + u  \frac{\partial}{\partial x } u + v  \frac{\partial}{\partial y} u -  \nu  \frac{\partial^2}{\partial y^2} \ u  =  ( -  \frac{\partial P}{\partial x} )$$

$$\frac{\partial u}{\partial x} + \frac{\partial v}{\partial y} = 0$$

\subsubsection{Blasius Similarity Solution}

$$f'=0\ at\ \eta=0$$
$$f=0\ at\ \eta=0$$
$$f' =2 \ at \ \eta\rightarrow \infty$$
$$f''f+f'''=0$$

$$u = \frac{u_\infty}{2} f'$$
$$\eta  = \frac{1}{2}y \sqrt{\frac{u_\infty}{\nu x}} $$
[not derived previously:]
$$\tau_x = \mu \frac{\partial u}{\partial y} = \rho \nu \frac{u_\infty}{4} \sqrt{\frac{u_\infty}{\nu x}} f''$$

\begin{verbatim}
Welty, J., Rorrer, G. L., & Foster, D. G. (2014). 
Fundamentals of momentum, heat, and mass transfer. John Wiley & Sons.
\end{verbatim}

\subsection{Integral Solution by Theodore Von Karman}
Final form of Von Karman equation:

$$ \frac{\tau_x}{\rho} = ( \frac{\partial u_\infty}{\partial x}) \int_0^{\delta_p} ( u_\infty - u ) \ dy  + \frac{\partial}{\partial x}   \int_0^{\delta_p} u( u_\infty - u) \ dy $$

\section{Solutions to BL}

\subsection{Von Karman Results (approximate solution)}
$$ \frac{\delta_p (x)}{x} =   \frac{4.64095 }{\sqrt{Re_x} }$$


$$C_{fx} \equiv \frac{0.6464}{ \sqrt{Re_x}}  $$

$$C_{fL} \equiv \frac{1.2928 }{\sqrt{Re_L}}  $$

\subsection{Blasius Results (Exact solution) }


$$ \frac{\delta_p (x)}{x} =   \frac{5}{\sqrt{Re_x} }$$


$$C_{fx} \equiv \frac{0.664}{ \sqrt{Re_x}}  $$

$$C_{fL} \equiv \frac{1.328}{\sqrt{Re_L}}  $$

7.2\% off for BL thickness, and 3\% off for skin friction coeff.
Pretty good!

This shows that Von Karman method is pretty good, if you can't use Blasius results 

\begin{verbatim}
Welty, J., Rorrer, G. L., & Foster, D. G. (2014). 
Fundamentals of momentum, heat, and mass transfer. John Wiley & Sons.
\end{verbatim}

\part{Energy Equations}


$$h \equiv e + PV$$

Enthalpy per unit mass (Specific enthalpy)
P is pressure
V is specific volume
e is specific internal energy

In thermodynamics, U is often used for internal energy, but we avoid using that because it gets confused with u which is x velocity. So we use e instead.

Enthalpy Balance

\begin{equation}
\frac{\partial }{\partial t} h + u \frac{\partial}{\partial x} h + v \frac{\partial }{\partial y} h + w \frac{\partial }{\partial z} h  = conduction + heat \ generation + dissipation+radiation
\end{equation}

neglect heat generation and dissipation terms. Also neglect radiation heat transfer

Fourier's law

$$q''_{cond} = - k \nabla T$$

Conduction term:

$$conduction = -\nabla q''_{cond}$$

Heat flux = $q''_{cond}$ ; 

Heat flux is heat energy Transferred per unit area 

$$\frac{\partial }{\partial t} h + u \frac{\partial}{\partial x} h + v \frac{\partial }{\partial y} h + w \frac{\partial }{\partial z} h  = -\nabla q''_{cond} $$ 


$$\frac{\partial }{\partial t} h + u \frac{\partial}{\partial x} h + v \frac{\partial }{\partial y} h + w \frac{\partial }{\partial z} h  = -\nabla (- k \nabla T) $$ 

$$\frac{\partial }{\partial t} h + u \frac{\partial}{\partial x} h + v \frac{\partial }{\partial y} h + w \frac{\partial }{\partial z} h  = \nabla ( k \nabla T) $$ 

heat conduction coeffcient is isotropic (it's a scalar), we also assume k does not change with x,y or z

$$\frac{\partial }{\partial t} h + u \frac{\partial}{\partial x} h + v \frac{\partial }{\partial y} h + w \frac{\partial }{\partial z} h  = k \nabla^2 T $$ 

$$\Delta h = \rho c_p  \Delta T$$
$$h-h_{ref} = \rho c_p  (T-T_{ref})$$
$$h = \rho c_p  (T-T_{ref}) + h_{ref}$$

We note, reference temperature and enthalpy are constant with time, x y z etc.


$$\frac{\partial }{\partial t} \rho c_p T + u \frac{\partial}{\partial x} \rho c_p T + v \frac{\partial }{\partial y} \rho c_p T + w \frac{\partial }{\partial z} \rho c_p T  = k \nabla^2 T $$ 

We assume, volumetric heat capacity does not change with x,y or z and time

$$\rho c_p  = volumetric \ heat \ capacity = \frac{kg}{m^3} \frac{J}{kg \bullet k}$$

$$\frac{\partial }{\partial t} \rho c_p T + u \frac{\partial}{\partial x} \rho c_p T + v \frac{\partial }{\partial y} \rho c_p T + w \frac{\partial }{\partial z} \rho c_p T  = k \nabla^2 T $$ 

$$\frac{\partial }{\partial t} T + u \frac{\partial}{\partial x} T + v \frac{\partial }{\partial y} \ T + w \frac{\partial }{\partial z} T  = \frac{k}{\rho c_p} \nabla^2 T $$ 

$$\frac{\partial }{\partial t} T + u \frac{\partial}{\partial x} T + v \frac{\partial }{\partial y} \ T + w \frac{\partial }{\partial z} T  = \alpha \nabla^2 T $$ 

$$\alpha = \frac{k}{\rho c_p}$$

$$\frac{\partial }{\partial t} T + u \frac{\partial}{\partial x} T + v \frac{\partial }{\partial y} \ T + w \frac{\partial }{\partial z} T  = \alpha ( \frac{\partial^2}{\partial x^2} T +  \frac{\partial^2 }{\partial y^2} \ T +  \frac{\partial^2 }{\partial z^2} T) $$ 

\part{Thermal Laminar BL Equations }

First, we assume, 2D, so z derivatives are zero.

$$\frac{\partial }{\partial t} T + u \frac{\partial}{\partial x} T + v \frac{\partial }{\partial y} \ T + w \frac{\partial }{\partial z} T  = \alpha ( \frac{\partial^2}{\partial x^2} T +  \frac{\partial^2 }{\partial y^2} \ T +  \frac{\partial^2 }{\partial z^2} T) $$ 


$$\frac{\partial }{\partial t} T + u \frac{\partial}{\partial x} T + v \frac{\partial }{\partial y} \ T  = \alpha ( \frac{\partial^2}{\partial x^2} T +  \frac{\partial^2 }{\partial y^2} \ T ) $$ 

We need to introduce relative temperature  

$$T_{rel} = T-T_s$$

$$\frac{\partial}{\partial x} (T-T_s) =\frac{\partial}{\partial x} (T) $$

so long as $T_s$ is constant

$$\frac{\partial }{\partial t} (T-T_s) + u \frac{\partial}{\partial x} (T-T_s) + v \frac{\partial }{\partial y} (T-T_s)  = \alpha ( \frac{\partial^2}{\partial x^2} (T-T_s) +  \frac{\partial^2 }{\partial y^2} (T-T_s) ) $$ 

Let's see which terms we can neglect

$$\theta = \frac{T-T_s}{T_\infty	-T_s}$$
$$u^* = \frac{u}{u_\infty}$$

Let's nondimensionaise temrperature:

$$\frac{\partial }{\partial t} \theta(T_\infty-T_s) + u \frac{\partial}{\partial x} \theta(T_\infty-T_s)  + v \frac{\partial }{\partial y} \theta(T_\infty-T_s)   = \alpha ( \frac{\partial^2}{\partial x^2} \theta(T_\infty-T_s)  +  \frac{\partial^2 }{\partial y^2} \theta(T_\infty-T_s) ) $$ 

Divide throughout,

$$\frac{\partial }{\partial t} \theta + u \frac{\partial}{\partial x} \theta + v \frac{\partial }{\partial y} \theta   = \alpha ( \frac{\partial^2}{\partial x^2} \theta  +  \frac{\partial^2 }{\partial y^2} \theta ) $$ 

$$x^* = \frac{x}{L}$$

$$y^* = \frac{y}{\delta_t}$$

Note: in general, thermal boundary layer is different from momentum boundary layer


$$\frac{\partial }{\partial t} \theta + u \frac{1}{L} \frac{\partial}{\partial x^*} \theta + v \frac{1}{\delta_t} \frac{\partial }{\partial y^*} \theta   = \alpha ( \frac{1}{L^2} \frac{\partial^2}{\partial (x^*)^2} \theta  +  \frac{1}{\delta_t^2} \frac{\partial^2 }{\partial (y^*)^2} \theta ) $$ 

$$u^* = \frac{u}{u_\infty}$$


$$\frac{\partial }{\partial t} \theta + u^* \frac{u_\infty}{L} \frac{\partial}{\partial x^*} \theta + v \frac{1}{\delta_t} \frac{\partial }{\partial y^*} \theta   = \alpha ( \frac{1}{L^2} \frac{\partial^2}{\partial (x^*)^2} \theta  +  \frac{1}{\delta_t^2} \frac{\partial^2 }{\partial (y^*)^2} \theta ) $$ 

We assume no natural convection

$$\frac{\partial }{\partial t} \theta + u^* \frac{u_\infty}{L} \frac{\partial}{\partial x^*} \theta + v \frac{1}{\delta_t} \frac{\partial }{\partial y^*} \theta   = \alpha ( \frac{1}{L^2} \frac{\partial^2}{\partial (x^*)^2} \theta  +  \frac{1}{\delta_t^2} \frac{\partial^2 }{\partial (y^*)^2} \theta ) $$ 

If we were to use momentum nondimensionalisation:

$$y^* = \frac{y}{\delta_p}$$
$$\frac{\partial u}{\partial x} + \frac{\partial v}{\partial y} = 0$$

$$ \frac{u_\infty}{L} \frac{\partial u^*}{\partial x^*} + \frac{v_c}{\delta_p} \frac{\partial v^*}{\partial y^*} = 0$$

$$ \frac{u_\infty \delta_p}{v_c L} \frac{\partial u^*}{\partial x^*} +  \frac{\partial v^*}{\partial y^*} = 0$$

$$\frac{u_\infty \delta_p}{v_c L} = \mathcal{O}(1)$$

$$\frac{u_\infty \delta_p}{L} = \mathcal{O}(1) v_c$$

we can just set:

$$v_c = \frac{u_\infty \delta_p}{L}\mathcal{O}(1)$$

so that

$$v^* = \frac{v}{v_c} = \mathcal{O}(1)$$

or else

$$v_c = \frac{u_\infty \delta_p}{L}$$

So we substitute this back in:

$$\frac{\partial }{\partial t} \theta + u^* \frac{u_\infty}{L} \frac{\partial}{\partial x^*} \theta + \frac{u_\infty \delta_p}{L \delta_t} v^* \frac{\partial }{\partial y^*} \theta   = \alpha ( \frac{1}{L^2} \frac{\partial^2}{\partial (x^*)^2} \theta  +  \frac{1}{\delta_t^2} \frac{\partial^2 }{\partial (y^*)^2} \theta ) $$ 

$$t = \frac{t^*}{t_c}$$

if you talk about hydrodynamic timescales:
$$t_c = \frac{L}{u_\infty}$$

we have to use thermal BL timescales...

For simplicity, we may not want to consider transient timescales in flat plate BL analysis

We assume steady state

$$u^* \frac{u_\infty}{L} \frac{\partial}{\partial x^*} \theta + \frac{u_\infty \delta_p}{L \delta_t} v^* \frac{\partial }{\partial y^*} \theta   = \alpha ( \frac{1}{L^2} \frac{\partial^2}{\partial (x^*)^2} \theta  +  \frac{1}{\delta_t^2} \frac{\partial^2 }{\partial (y^*)^2} \theta ) $$ 

$$u^* \frac{\partial}{\partial x^*} \theta + \frac{\delta_p}{ \delta_t} v^* \frac{\partial }{\partial y^*} \theta   = \alpha \frac{L}{u_\infty} ( \frac{1}{L^2} \frac{\partial^2}{\partial (x^*)^2} \theta  +  \frac{1}{\delta_t^2} \frac{\partial^2 }{\partial (y^*)^2} \theta ) $$ 

We can make this observation:

$$L^2 >> \delta_t^2$$

$$u^* \frac{\partial}{\partial x^*} \theta + \frac{\delta_p}{ \delta_t} v^* \frac{\partial }{\partial y^*} \theta   = \alpha \frac{L}{u_\infty} (\frac{1}{\delta_t^2} \frac{\partial^2 }{\partial (y^*)^2} \theta ) $$ 

assume steady state, and performing BL analysis:
Also, no dissipation or heat generation

$$ u \frac{\partial}{\partial x} T + v \frac{\partial }{\partial y} \ T  = \alpha (\frac{\partial^2 }{\partial y^2} \ T ) $$ 

\part{Solution Procedures - Constant Temp Forced Conv BL}


$$ u \frac{\partial}{\partial x} T + v \frac{\partial }{\partial y} T  = \alpha \frac{\partial^2 }{\partial y^2}T  $$ 
$$0 = - \frac{\partial P}{\partial y} $$

$$u  \frac{\partial}{\partial x } u + v  \frac{\partial}{\partial y} u -  \nu  \frac{\partial^2}{\partial y^2} \ u  =  ( -  \frac{\partial P}{\partial x} )$$

$$\frac{\partial u}{\partial x} + \frac{\partial v}{\partial y} = 0$$

Solution Procedures:
\begin{itemize}
\item Similarity Solution (Blasius Style) [GOLD STANDARD]
\item Integral Solution (Von Karman Style)
\item Computational Fluid Dynamics (MultiPhysics) 
\end{itemize}

\section{back to similarity Solution}

In case of no pressure gradient,
$$ u \frac{\partial}{\partial x} T + v \frac{\partial }{\partial y} T  = \alpha \frac{\partial^2 }{\partial y^2}T  $$ 

$$u  \frac{\partial}{\partial x } u + v  \frac{\partial}{\partial y} u  =  \nu  \frac{\partial^2}{\partial y^2} \ u$$

$$\frac{\partial u}{\partial x} + \frac{\partial v}{\partial y} = 0$$

And nondimensionalising 
$$f'=0\ at\ \eta=0$$
$$f=0\ at\ \eta=0$$
$$f' =2 \ at \ \eta\rightarrow \infty$$
$$f''f+f'''=0$$

$$u = \frac{u_\infty}{2} f'$$
$$\eta  = \frac{1}{2}y \sqrt{\frac{u_\infty}{\nu x}} $$
[not derived previously:]
$$\tau_x = \mu \frac{\partial u}{\partial y} = \rho \nu \frac{u_\infty}{4} \sqrt{\frac{u_\infty}{\nu x}} f''$$

\subsubsection{Reynold's analogy}

Hey these equations look so similar
$$ u \frac{\partial}{\partial x} T + v \frac{\partial }{\partial y} T  = \alpha \frac{\partial^2 }{\partial y^2}T  $$ 

$$u  \frac{\partial}{\partial x } u + v  \frac{\partial}{\partial y} u  =  \nu  \frac{\partial^2}{\partial y^2} \ u$$

Let's copy/paste the solution... (under certain constraints)

$$\nu = \alpha $$

$$Pr = \frac{\nu}{\alpha} = 1$$

If you wanna change up the equations:

$$ u \frac{\partial}{\partial x} T + v \frac{\partial }{\partial y} T  = \frac{\nu}{Pr} \frac{\partial^2 }{\partial y^2}T  $$ 

If Pr=1,

$$ u \frac{\partial}{\partial x} T + v \frac{\partial }{\partial y} T  = \nu \frac{\partial^2 }{\partial y^2}T  $$ 

Use the relative temperature trick, and change the variable to $\theta$

$$ u \frac{\partial}{\partial x} \theta + v \frac{\partial }{\partial y} \theta  = \nu \frac{\partial^2 }{\partial y^2} \theta  $$ 
If we nondimensionalise the momentum equation

$$u  \frac{\partial}{\partial x } u^* u_\infty + v  \frac{\partial}{\partial y} u^* u_\infty  =  \nu  \frac{\partial^2}{\partial y^2} u^* u_\infty$$

$$u  \frac{\partial}{\partial x } u^*  + v  \frac{\partial}{\partial y} u^*   =  \nu  \frac{\partial^2}{\partial y^2} u^* $$


We already proved we can reduce the momentum BL equations as follows:
$$f''f+f'''=0$$

And we note that we can set:

$$ \theta  = u^*$$
Note that
$$u = \frac{u_\infty}{2} f'$$

$$u^* = \frac{f'}{2}$$

Under Reynold's analogy we can say:

$$\theta = \frac{f'}{2}$$

We can pretty much adapt the Blasius solution to thermal boundary layer... (Reynold's analogy under Pr=1)

We assume that the nondimensional temperature profile and nondimensional BL profile (and BCs) are exactly equal.

Based on that, we can get the temperature profile and heat flux

$$\theta = \frac{f'}{2}$$
$$T-T_s = (T_\infty -T_s) \frac{f'}{2}$$

What about heat flux:

$$q'' = - k \frac{\partial T}{\partial y}$$

note:

$$\frac{\partial u}{\partial y} = \frac{u_\infty}{4} \sqrt{\frac{u_\infty}{\nu x}} f''$$

$$\frac{\partial u^* u_\infty}{\partial y} = \frac{u_\infty}{4} \sqrt{\frac{u_\infty}{\nu x}} f''$$

$$\frac{\partial u^*}{\partial y} = \frac{1}{4} \sqrt{\frac{u_\infty}{\nu x}} f''$$

under reynold's analogy
$$u^* = \theta $$

$$\frac{\partial \theta}{\partial y} = \frac{1}{4} \sqrt{\frac{u_\infty}{\nu x}} f''$$

$$\frac{\partial \theta}{\partial y} = \frac{1}{4} \sqrt{\frac{u_\infty}{\nu x}} f''$$
$$\frac{\partial \theta}{\partial y} = \frac{1}{4} \sqrt{\frac{u_\infty x}{\nu x^2}} f''$$

$$\frac{\partial \theta}{\partial y} = \frac{1}{4} \sqrt{Re_x} \frac{1}{x} f''$$

$$\frac{\partial (T-T_s)}{\partial y} = \frac{T_\infty - T_s}{4} \sqrt{Re_x} \frac{1}{x} f''$$

$$\frac{\partial T}{\partial y} = \frac{T_\infty - T_s}{4} \sqrt{Re_x} \frac{1}{x} f''$$

Heat flux:

$$q'' = - k \frac{\partial T}{\partial y}$$

$$q'' = - \frac{k}{x} \frac{T_\infty - T_s}{4} \sqrt{Re_x} f''$$

Heat trf Coefficient and Nu

$$q''= - h (T_\infty - T_s)$$

Substitute

$$h (T_\infty - T_s) =  \frac{k}{x} \frac{T_\infty - T_s}{4} \sqrt{Re_x} f''$$
$$h  =  \frac{k}{x} \frac{f''}{4} \sqrt{Re_x}$$

Define Nusselt Number:

$$Nu_x \equiv \frac{hx}{k}$$

$$Nu_x = \frac{f''}{4} \sqrt{Re_x}$$

What is the value of f'' (look at textbook, or solve it yourself)

at y=0
$$f'' = 1.328$$

$$Nu_x = \frac{1.328}{4} \sqrt{Re_x} = 0.332 \sqrt{Re_x} $$

\section{What if Pr is not 1?}

\subsection{Similarity Solution by Pohlhausen}
\begin{verbatim}
Bejan, A. (2013). Convection heat transfer. John wiley & sons.
Welty, J., Rorrer, G. L., & Foster, D. G. (2014). 
Fundamentals of momentum, heat, and mass transfer. John Wiley & Sons.
\end{verbatim}


From Blasius Solution

And nondimensionalising 
$$f'=0\ at\ \eta=0$$
$$f=0\ at\ \eta=0$$
$$f' =2 \ at \ \eta\rightarrow \infty$$
$$f''f+f'''=0$$

$$u = \frac{u_\infty}{2} f'$$
Some useful derivatives previously derived (\cite{bejan2013convection},\cite{welty2014fundamentals})

From hydrodynamic BL:
$$\frac{\partial}{\partial y^*} \eta =  \frac{\partial}{\partial y^*} \frac{1}{2} \sqrt{Re_\delta} \frac{y^*}{\sqrt{x^*}} = \frac{\eta}{y^*}$$


$$ \frac{\partial}{\partial x^*} \eta =  \frac{\partial}{\partial x^*}\frac{1}{2} \sqrt{Re_\delta} \frac{y^*}{\sqrt{x^*}}  $$



$$ u \frac{\partial}{\partial x} T + v \frac{\partial }{\partial y} T  = \alpha \frac{\partial^2 }{\partial y^2}T  $$ 

We nondimensionalised the energy equation during scaling analysis...

$$u^* \frac{\partial}{\partial x^*} \theta + \frac{\delta_p}{ \delta_t} v^* \frac{\partial }{\partial y^*} \theta   = \alpha \frac{L}{u_\infty} (\frac{1}{\delta_t^2} \frac{\partial^2 }{\partial (y^*)^2} \theta ) $$ 

We just need to convert it to include similarity variables...

We start by making substitutions
$$u^* = \frac{f'}{2}$$

$$v= \frac{1}{2} \sqrt{\frac{\nu u_\infty}{x}}(\eta f' -f)$$

$$v_c = \frac{u_\infty \delta_p}{L}$$

$$v^* \frac{u_\infty \delta_p}{L} = -\frac{1}{2} \sqrt{\frac{\nu u_\infty}{x}}(\eta f' -f)$$

$$v^*  = \frac{1}{2} \frac{L}{\delta_p} \sqrt{\frac{\nu }{x u_\infty}}(\eta f' -f)$$

$$v^*  = \frac{1}{2} \frac{L}{\delta_p} Re_x^{-\frac{1}{2}} (\eta f' -f)$$


subs into:

$$u^* \frac{\partial}{\partial x^*} \theta + \frac{\delta_p}{ \delta_t} v^* \frac{\partial }{\partial y^*} \theta   = \alpha \frac{L}{u_\infty} (\frac{1}{\delta_t^2} \frac{\partial^2 }{\partial (y^*)^2} \theta ) $$ 

$$\frac{f'}{2} \frac{\partial}{\partial x^*} \theta + \frac{\delta_p}{ \delta_t} \left[ -\frac{1}{2} \frac{L}{\delta_p} Re_x^{\frac{1}{2}} (\eta f' -f) \right] \frac{\partial }{\partial y^*} \theta   = \alpha \frac{L}{u_\infty} (\frac{1}{\delta_t^2} \frac{\partial^2 }{\partial (y^*)^2} \theta ) $$ 


$$\frac{f'}{2} \frac{\partial}{\partial x^*} \theta + \frac{L}{ \delta_t} \left[ -\frac{1}{2}  Re_x^{\frac{1}{2}} (\eta f' -f) \right] \frac{\partial }{\partial y^*} \theta   = \alpha \frac{L}{u_\infty} (\frac{1}{\delta_t^2} \frac{\partial^2 }{\partial (y^*)^2} \theta ) $$ 

Now we need to substitute out $y^*$ and $x^*$ for $\eta$

$$y^* = \frac{y}{\delta_t}$$

$$x^* = \frac{x}{L}$$

we note $x^*$ is the same definition as the nondimensionalised $x^*$ used in deriving hydrodynamic BL, so we use it again...

So we can use:

$$ \frac{\partial}{\partial x^*} \eta =  \frac{\partial}{\partial x^*}\frac{1}{2} \sqrt{Re_\delta} \frac{y^*}{\sqrt{x^*}}  $$

$$ = \frac{1}{2} \sqrt{Re_\delta} \frac{y^*}{\sqrt{x^*}} \frac{-1}{2x^*} = \frac{\eta}{-2x^*} $$

Replace left most term...

$$\frac{\partial}{\partial x^*} = \frac{\partial \eta}{\partial x^*} \frac{\partial }{\partial \eta} = - \frac{\eta}{2x^*} \frac{\partial }{\partial \eta}  $$

$$\frac{f'}{2} \frac{\partial}{\partial x^*} \theta + \frac{L}{ \delta_t} \left[ \frac{1}{2}  Re_x^{-\frac{1}{2}} (\eta f' -f) \right] \frac{\partial }{\partial y^*} \theta   = \alpha \frac{L}{u_\infty} (\frac{1}{\delta_t^2} \frac{\partial^2 }{\partial (y^*)^2} \theta ) $$ 

$$\frac{f'}{2} ( - \frac{\eta}{2x^*}) \frac{\partial }{\partial \eta} \theta + \frac{L}{ \delta_t} \left[ \frac{1}{2}  Re_x^{-\frac{1}{2}} (\eta f' -f) \right] \frac{\partial }{\partial y^*} \theta   = \alpha \frac{L}{u_\infty} (\frac{1}{\delta_t^2} \frac{\partial^2 }{\partial (y^*)^2} \theta ) $$ 

$$\frac{f'}{2} ( - \frac{\eta}{2x^*}) \theta' + \frac{L}{ \delta_t} \left[ \frac{1}{2}  Re_x^{-\frac{1}{2}} (\eta f' -f) \right] \frac{\partial }{\partial y^*} \theta   = \alpha \frac{L}{u_\infty} (\frac{1}{\delta_t^2} \frac{\partial^2 }{\partial (y^*)^2} \theta ) $$ 

$$\frac{f'}{2} ( - \frac{\eta}{2x^*}) \theta' + L \left[ \frac{1}{2}  Re_x^{-\frac{1}{2}} (\eta f' -f) \right] \frac{\partial }{\partial y} \theta   = \alpha \frac{L}{u_\infty} ( \frac{\partial^2 }{\partial y^2} \theta ) $$ 


What about the y terms?

$$\eta = \frac{1}{2} \frac{y}{x} (Re_x)^{\frac{1}{2}} = \frac{1}{2} y \sqrt{\frac{u_\infty}{\nu x}}$$

$$\frac{\partial}{\partial y} = \frac{\partial \eta}{\partial y} \frac{\partial }{\partial \eta}$$

$$\frac{\partial \eta}{\partial y}= \frac{1}{2} \sqrt{\frac{u_\infty}{\nu x}} = \frac{\eta}{y}$$

For RHS...

$$\frac{\partial ^2 \theta}{\partial y^2} = \frac{\partial}{\partial y} (\frac{\partial  \eta}{\partial y} \frac{\partial \theta}{\partial \eta}) = \frac{\partial}{\partial y} (\frac{\partial  \eta}{\partial y} \theta') $$ 

$$ = \frac{\partial \theta' }{\partial y} \frac{\partial \eta }{\partial y} + \frac{\partial^2 \eta}{\partial y^2} \theta'$$

$$\frac{\partial^2 \eta}{\partial y^2} = \frac{\partial }{\partial y} (\frac{\eta}{y}) = 0 $$

$$\frac{\partial ^2 \theta}{\partial y^2} = \frac{\partial \theta' }{\partial y} \frac{\partial \eta }{\partial y} =  \frac{\partial \eta}{\partial y} \frac{\partial \theta'}{\partial \eta} \frac{\partial \eta }{\partial y} = \theta'' (\frac{\eta^2}{y^2})$$ 

Substitution back in:

$$\frac{f'}{2} ( - \frac{\eta}{2x^*}) \theta' + L \left[ -\frac{1}{2}  Re_x^{-\frac{1}{2}} (\eta f' -f) \right] \frac{\eta }{ y} \theta'   = \alpha \frac{L}{u_\infty} ( \frac{\eta^2 }{y^2} \theta'' ) $$ 

tidying up:


$$( - \frac{\eta f'}{4x^*}) \theta' + L \left[ \frac{1}{2}  Re_x^{-\frac{1}{2}} (\eta f' -f) \right] \frac{\eta }{ y} \theta'   = \alpha \frac{L}{u_\infty} ( \frac{\eta^2 }{y^2} \theta'' ) $$ 

$$( - \frac{\eta f'}{4x^*}) \theta'   +\frac{1}{2} \frac{\eta }{ y} L   Re_x^{-\frac{1}{2}} (\eta f' -f) \theta'    = \alpha \frac{L}{u_\infty} ( \frac{\eta^2 }{y^2} \theta'' ) $$ 

$$( - \frac{\eta f'}{4x^*}) \theta'   +\frac{1}{2} \frac{\eta }{ y} L   Re_x^{-\frac{1}{2}} (\eta f'  \theta' -f  \theta')    = \alpha \frac{L}{u_\infty} ( \frac{\eta^2 }{y^2} \theta'' ) $$ 

$$( - \frac{\eta f'}{4x^*}) \theta'  +\frac{1}{2} \frac{\eta }{ y} L   Re_x^{-\frac{1}{2}} \eta f'  \theta' - \frac{1}{2} \frac{\eta }{ y} L   Re_x^{-\frac{1}{2}}  f  \theta'   = \alpha \frac{L}{u_\infty} ( \frac{\eta^2 }{y^2} \theta'' ) $$ 

$$\theta' \eta f' ( - \frac{1}{4x^*} +\frac{1}{2} \frac{\eta }{ y} L   Re_x^{-\frac{1}{2}} )   - \frac{1}{2} \frac{\eta }{ y} L   Re_x^{-\frac{1}{2}}  f  \theta'   = \alpha \frac{L}{u_\infty} ( \frac{\eta^2 }{y^2} \theta'' ) $$ 

simplifying...

$$\theta' \eta f' ( - \frac{1}{4x^*} + \frac{1}{4} \sqrt{\frac{u_\infty}{\nu x}}  L   Re_x^{-\frac{1}{2}} )   - \frac{1}{2} \frac{\eta }{ y} L   Re_x^{-\frac{1}{2}}  f  \theta'   = \alpha \frac{L}{u_\infty} ( \frac{\eta^2 }{y^2} \theta'' ) $$ 


$$\theta' \eta f' ( - \frac{1}{4x^*} + \frac{1}{4 x} \sqrt{Re_x}  L   Re_x^{-\frac{1}{2}} )   - \frac{1}{2} \frac{\eta }{ y} L   Re_x^{-\frac{1}{2}}  f  \theta'   = \alpha \frac{L}{u_\infty} ( \frac{\eta^2 }{y^2} \theta'' ) $$ 

$$\theta' \eta f' ( - \frac{1}{4x^*} + \frac{L}{4 x}   )   - \frac{1}{2} \frac{\eta }{ y} L   Re_x^{-\frac{1}{2}}  f  \theta'   = \alpha \frac{L}{u_\infty} ( \frac{\eta^2 }{y^2} \theta'' ) $$ 

$$\theta' \eta f' ( - \frac{1}{4x^*} + \frac{1}{4 x^*}   )  - \frac{1}{2} \frac{\eta }{ y} L   Re_x^{-\frac{1}{2}}  f  \theta'   = \alpha \frac{L}{u_\infty} ( \frac{\eta^2 }{y^2} \theta'' ) $$ 

$$ -\frac{1}{2} \frac{\eta }{ y} L   Re_x^{-\frac{1}{2}}  f  \theta'   = \alpha \frac{L}{u_\infty} ( \frac{\eta^2 }{y^2} \theta'' ) $$

 $$- \frac{1}{2} \frac{1 }{2} \frac{1}{x} Re_x^{\frac{1}{2}} L   Re_x^{-\frac{1}{2}}  f  \theta'   = \alpha \frac{L}{u_\infty} ( \frac{\eta^2 }{y^2} \theta'' ) $$


 $$-\frac{1}{4} \frac{1}{x^*}   f  \theta'   = \alpha \frac{L}{u_\infty} ( \frac{\eta^2 }{y^2} \theta'' ) $$
 
  $$- \frac{1}{4} \frac{1}{x^*}   f  \theta'   = \alpha \frac{L}{u_\infty} ( \frac{1}{4} \frac{Re_x}{x^2} \theta'' ) $$

  $$-\frac{1}{4} \frac{1}{x^*}   f  \theta'   = \alpha \frac{1}{u_\infty} ( \frac{1}{4} \frac{Re_x}{x x^*} \theta'' ) $$

  $$ -\frac{1}{4}    f  \theta'   = \alpha \frac{1}{u_\infty} ( \frac{1}{4} \frac{Re_x}{x } \theta'' ) $$

$$ - f  \theta'   = \alpha \frac{1}{u_\infty} ( \frac{Re_x}{x } \theta'' ) $$
  
$$  -   f  \theta'   = \alpha \frac{1}{u_\infty} ( \frac{u_\infty x}{x \nu } \theta'' ) $$

$$  -   f  \theta'   = \frac{ \alpha}{\nu} \theta'' $$

$$  -   f  \theta'   = \frac{1}{Pr} \theta'' $$

$$\theta'' + Pr \ f \theta' = 0$$

\subsection{Similarity Solution BCs}

2 Bcs:

$$y \rightarrow \infty \ ; \ T \rightarrow T_\infty$$

$$\eta \rightarrow \infty \ ; \ \theta \rightarrow 1$$
Reminder

$$\theta = \frac{T-T_s}{T_\infty - T_s}$$

$$y=0 ; T=T_s$$
$$\eta = 0 \  ; \theta = 0 $$

\subsection{solving the equation}

$$\theta'' + Pr \ f \theta' = 0$$

$$ \frac{\partial  \theta'}{\partial \eta} + Pr \ f \theta' = 0$$

$$ \frac{\partial  \theta'}{\partial \eta} = - Pr \ f \theta' $$

$$  \frac{\partial  \theta'}{\partial \eta} = - Pr \ f \theta' $$

$$ \frac{1}{\theta'} \frac{\partial  \theta'}{\partial \eta} = - Pr \ f$$

$$ \int_0^\infty  \frac{1}{\theta'} \frac{\partial  \theta'}{\partial \eta} d \eta = -  \int_0^\infty  Pr \ f d\eta $$

$$ \int_{\theta'|_{\eta=0}}^{\theta'|_{\eta=\infty}} \frac{1}{\theta'}  d \theta' = -  \int_0^\infty  Pr \ f (\eta) d\eta $$

$$ \log_e (\frac{\theta'|_{\eta=\infty}}{\theta'|_{\eta=0}}) = -  \int_0^\infty  Pr \ f (\eta) d\eta $$

$$ \frac{\theta'|_{\eta=\infty}}{\theta'|_{\eta=0}} = \exp ( -  \int_0^\infty  Pr \ f (\eta) d\eta) $$

What if $\eta$ is not infinity,

$$ \int_{\theta'|_{\eta'=0}}^{\theta'|_{\eta'=\eta}} \frac{1}{\theta'}  d \theta' = -  \int_0^\eta  Pr \ f (\eta') d\eta' $$

$$ \log_e (\frac{\theta'|_{\eta'=\infty}}{\theta'|_{\eta=0}}) = -  \int_0^\infty  Pr \ f (\eta) d\eta $$

$$ \frac{\theta'|_{\eta'=\eta}}{\theta'|_{\eta=0}} = \exp ( -  \int_0^\eta  Pr \ f (\eta') d\eta') $$

$$ \theta'|_{\eta} =  \theta'|_{\eta=0}\exp ( -  \int_0^\eta  Pr \ f (\eta') d\eta') $$

Integrate again,

$$ \int_0^\eta \theta' (\eta') d\eta' = \int_0^\eta  \theta'|_{\eta'=0}\exp ( -  \int_0^\eta  Pr \ f (\eta') d\eta') d\eta' $$

$$ \theta (\eta) = \int_0^\eta  \theta'|_{\eta=0}\exp ( -  \int_0^\eta  Pr \ f (\eta') d\eta') d\eta' $$

use other conventional dummy variables to avoid confusion

$$ \theta (\eta) = \int_{\gamma=0}^{\gamma=\eta}  \theta'|_{\eta'=0}\exp ( -  \int_{\beta=0}^{\beta=\eta}  Pr \ f (\beta) d\beta) \gamma $$

One problem: we don't know what $\theta'|_{\eta'=0}$ is

we use the boundary condition:

$$\theta=1 \ at \ \eta = infinity$$

the dummy variable representing $\eta$ is $\gamma$


$$ \theta (\eta=\infty) = \int_{\gamma=0}^{\gamma=\infty}  \theta'|_{\eta=0}\exp ( -  \int_{\beta=0}^{\beta=\eta}  Pr \ f (\beta) d\beta) \gamma $$


$$ 1 = \int_{\gamma=0}^{\gamma=\infty}  \theta'|_{\eta=0}\exp ( -  \int_{\beta=0}^{\beta=\eta}  Pr \ f (\beta) d\beta) \gamma $$

$$ \theta'(0)^{-1} = \int_{\gamma=0}^{\gamma=\infty}  \exp ( -  \int_{\beta=0}^{\beta=\eta}  Pr \ f (\beta) d\beta) \gamma $$

$$ \theta'(0) = \left[ \int_{\gamma=0}^{\gamma=\infty}  \exp ( -  \int_{\beta=0}^{\beta=\eta}  Pr \ f (\beta) d\beta) \gamma \right] ^{-1}$$

using correct mathematical symbols:

$$ \theta'(0) = \left[ \int_{\gamma=0}^{\gamma=\infty}  \exp ( -  \int_{\beta=0}^{\beta=\gamma}  Pr \ f (\beta) d\beta) \gamma \right] ^{-1}$$

$$ \theta (\eta) = \theta'(0) \int_{\gamma=0}^{\gamma=\eta}    \exp ( -   Pr   \int_{\beta=0}^{\beta=\gamma}\ f (\beta) d\beta) \gamma $$


We want to find out heat flux...

$$q'' = - k \frac{\partial T}{\partial y}|_{y=0}$$

$$q'' = - k \frac{\partial (T-T_s)}{\partial y}|_{y=0}$$
$$q'' = - k (T_\infty-T_s) \frac{\partial \theta}{\partial y}|_{y=0}$$

$$h = \frac{q''}{-(T_\infty-T_s)} = k  \frac{\partial \theta}{\partial y}|_{y=0}$$

$$ \frac{h}{k}=   \frac{\partial \theta}{\partial y}|_{y=0}$$

$$ \frac{h}{k}=  \frac{\partial \eta}{\partial y} \frac{\partial \theta}{\partial \eta}|_{y=0}$$

$$ \frac{h}{k}=  \frac{ \eta}{ y} \frac{\partial \theta}{\partial \eta}|_{y=0}$$


$$ \frac{h}{k}=  \frac{1}{2} \sqrt{\frac{u_\infty}{\nu x}} \frac{\partial \theta}{\partial \eta}|_{y=0}$$


$$ \frac{h}{k}=  \frac{1}{2} \sqrt{\frac{u_\infty}{\nu x}} \theta' (0)$$

$$ \frac{h}{k}=  \frac{1}{2} \sqrt{\frac{u_\infty x}{\nu x^2}} \theta' (0)$$


$$ \frac{h x}{k}=  \frac{1}{2} \sqrt{Re_x} \theta' (0)$$

$$ Nu_x=  \frac{1}{2} \sqrt{Re_x} \theta' (0)$$

$$Nu_x = \frac{1}{2} \sqrt{Re_x}  \left[ \int_{\gamma=0}^{\gamma=\infty}  \exp ( -  \int_{\beta=0}^{\beta=\gamma}  Pr \ f (\beta) d\beta) \gamma \right] ^{-1}$$

In textbook expression:

$$Nu_x =\sqrt{Re_x}  \left[ \int_{\gamma=0}^{\gamma=\infty}  \exp ( -  \int_{\beta=0}^{\beta=\gamma}  \frac{1}{2}  Pr \ f (\beta) d\beta) \gamma \right] ^{-1}$$


What is the value of:

$$\left[ \int_{\gamma=0}^{\gamma=\infty}  \exp ( -  \int_{\beta=0}^{\beta=\gamma}  \frac{1}{2}  Pr \ f (\beta) d\beta) \gamma \right] ^{-1}$$

Pohlhausen worked it out for Pr$>$0.5:

$$\left[ \int_{\gamma=0}^{\gamma=\infty}  \exp ( -  \int_{\beta=0}^{\beta=\gamma}  \frac{1}{2}  Pr \ f (\beta) d\beta) \gamma \right] ^{-1} \approx  0.332 Pr^{\frac{1}{3}} $$

Liquid metals $Pr <0.1$

Air $Pr=0.72$

Water $Pr=7$

Pr number says that for liquids of same viscosity, a higher prandtl number means lower  thermal diffusivity

Thermal diffusivity intuition, high thermal conductivity means high thermal diffusivity, all else constant.

$$Nu_x = 0.332 Re_x^{\frac{1}{2}}  Pr^{\frac{1}{3}} $$

$$Nu_L = \frac{1}{L} \int_0^L Nu_x dx$$
$$Nu_L = \frac{1}{L} \int_0^L 0.332 Re_x^{\frac{1}{2}}  Pr^{\frac{1}{3}}  dx$$
$$Nu_L = Pr^{\frac{1}{3}} \frac{1}{L}  \int_0^L 0.332 Re_x^{\frac{1}{2}}    dx$$
$$Nu_L = 0.664 Re_L^{\frac{1}{2}} Pr^{\frac{1}{3}}$$


Note: For the above solutions, use film temperature to evaluate thermal properties!

$$T_{film} = \frac{T_s + T_\infty}{2}$$

eg. $c_p, \mu$ etc..

\section{Integral Solutions to Thermal BL Constant Temp Flat Plate}

In the momentum BL, we considered a control volume in the BL to get:

$$ \frac{\tau_x}{\rho} = ( \frac{\partial u_\infty}{\partial x}) \int_0^{\delta_p} ( u_\infty - u ) \ dy  + \frac{\partial}{\partial x}   \int_0^{\delta_p} u( u_\infty - u) \ dy $$

Now for thermal BL, we can also consider control volume too (\cite{welty2014fundamentals}). 



We can consider again four sides of the control volume and perform an enthalpy balance, neglecting fluid kinetic energy:

$$\dot{Q} - \dot{W} = net \ enthalpy \ flows  \ out +  \Delta \ enthalpy \ w.r.t \ time$$

W is Work done by system, if Q+W then W is work done on system

$$\dot{Q} = q'' dA$$

$$q''_{wall} = - k \frac{\partial T}{\partial y} |_{y=0}$$

$$dA = l_z \Delta x$$

For conduction, ignore conduction within fluid. (important assumption)
2nd assumption, no work done W=0,

Let's consider the enthalpy flows and assume steady state..

Consider net enthalpy flows in:

neglecting kinetic energy

inflow of enthalpy through left

$$= \int_0^{\delta_t} \rho u h l_z dy |_{x}$$

$$= \int_0^{\delta_t} \rho u c_p (T-T_{ref}) l_z dy |_{x}$$


outflow of enthalpy through the right


$$= \int_0^{\delta_t} \rho u h l_z dy |_{x+\Delta x}$$
$$= \int_0^{\delta_t} \rho u c_p (T-T_{ref}) l_z dy |_{x+\Delta x}$$



inflow of enthalpy through the top

Specific enthalpy:

$$h_\infty = c_p (T_\infty - T_{ref})$$


mass flowrate into system:

$$\dot{m} = \int_0^{\delta_t} \rho u l_z dy |_{x+\Delta x} - \int_0^{\delta_t} \rho u l_z dy |_{x}$$

put this toegther to find enthalpy inflow through top: 

$$\dot{m} h_\infty = (\int_0^{\delta_t} \rho u l_z dy |_{x+\Delta x} - \int_0^{\delta_t} \rho u l_z dy |_{x}) c_p (T_\infty - T_{ref})$$

Let's subs everything back:


$$\dot{Q} - \dot{W} = net \ enthalpy \ flows  \ out +  \Delta \ enthalpy \ w.r.t \ time$$
$$- k \frac{\partial T}{\partial y} |_{y=0} l_z \Delta x - 0 = -(\int_0^{\delta_t} \rho u l_z dy |_{x+\Delta x} - \int_0^{\delta_t} \rho u l_z dy |_{x}) c_p (T_\infty - T_{ref}) $$

$$+ \int_0^{\delta_t} \rho u c_p (T-T_{ref}) l_z dy |_{x+\Delta x} -  \int_0^{\delta_t} \rho u c_p (T-T_{ref}) l_z dy |_{x}+  0$$



divide throughout by $\Delta x$ take limit $\Delta x \rightarrow 0$


$$- k \frac{\partial T}{\partial y} |_{y=0} l_z  = - \frac{1}{\Delta x} (\int_0^{\delta_t} \rho u l_z dy |_{x+\Delta x} - \int_0^{\delta_t} \rho u l_z dy |_{x}) c_p (T_\infty - T_{ref}) $$

$$+ \frac{1}{\Delta x} [ \int_0^{\delta_t} \rho u c_p (T-T_{ref}) l_z dy |_{x+\Delta x} -  \int_0^{\delta_t} \rho u c_p (T-T_{ref}) l_z dy |_{x}]$$

Assume $T_\infty $ constant with x


$$- k \frac{\partial T}{\partial y} |_{y=0} l_z   = - \frac{\partial}{\partial x} (\int_0^{\delta_t} \rho u l_z dy |) c_p (T_\infty - T_{ref}) $$

$$+ \frac{\partial}{\partial x} [ \int_0^{\delta_t} \rho u c_p (T-T_{ref}) l_z dy ]$$

Drop out $T_{ref}$ after bring $c_p T_\infty $ into integral:


$$- k \frac{\partial T}{\partial y} |_{y=0} l_z   = - \frac{\partial}{\partial x} (\int_0^{\delta_t} \rho u l_z dy  c_p T_\infty ) $$

$$+ \frac{\partial}{\partial x} [ \int_0^{\delta_t} \rho u c_p T l_z dy ]$$

Finally get rid of $l_z$

$$- k \frac{\partial T}{\partial y} |_{y=0}    = - \frac{\partial}{\partial x} (\int_0^{\delta_t} \rho u  dy  c_p T_\infty ) + \frac{\partial}{\partial x} [ \int_0^{\delta_t} \rho u c_p T  dy ]$$

$$- k \frac{\partial T}{\partial y} |_{y=0}    = - \frac{\partial}{\partial x} (\int_0^{\delta_t} \rho u  dy  c_p T_\infty ) + \frac{\partial}{\partial x} [ \int_0^{\delta_t} \rho u c_p T  dy ]$$


Tidy up equation:

$$- k \frac{\partial T}{\partial y} |_{y=0}    = \frac{\partial}{\partial x} [ \int_0^{\delta_t} \rho u c_p ( T -  T_\infty)  dy ]$$

y is integrated out so the integral on the RHS does not depend on y

$$- k \frac{\partial T}{\partial y} |_{y=0}    = \frac{d}{d x} [ \int_0^{\delta_t} \rho u c_p ( T -  T_\infty)  dy ]$$

You can even bring out $\rho c_p$:

$$- k \frac{\partial T}{\partial y} |_{y=0}    = \rho c_p \frac{d}{d x} [ \int_0^{\delta_t}  u  ( T -  T_\infty)  dy ]$$

\subsection{The Hardcore Integral Solution}
But in other textbooks, von karman's integral approach literally integrates the BL equations over y (\cite{bejan2013convection}). 


we start at:
$$ u \frac{\partial}{\partial x} T + v \frac{\partial }{\partial y} \ T  = \alpha (\frac{\partial^2 }{\partial y^2} \ T ) $$ 

and for we put back u and v into the differential using product rule. 

$$\frac{\partial}{\partial x}(uT) = u \frac{\partial}{\partial x}T + T \frac{\partial}{\partial x}u$$

$$u \frac{\partial}{\partial x}T =\frac{\partial}{\partial x}(uT)  - T \frac{\partial}{\partial x}u $$

$$\frac{\partial}{\partial y}(vT) = v \frac{\partial}{\partial y}T + T \frac{\partial}{\partial y}v$$

$$ v \frac{\partial}{\partial y}T =\frac{\partial}{\partial y}(vT) -T \frac{\partial}{\partial y}v  $$

subs back in:

$$ u \frac{\partial}{\partial x} T + v \frac{\partial }{\partial y} \ T  = \alpha (\frac{\partial^2 }{\partial y^2} \ T ) $$ 

$$ \frac{\partial}{\partial x}(uT)  - T \frac{\partial}{\partial x}u + \frac{\partial}{\partial y}(vT) -T \frac{\partial}{\partial y}v  = \alpha (\frac{\partial^2 }{\partial y^2} \ T ) $$ 

$$ \frac{\partial}{\partial x}(uT)   + \frac{\partial}{\partial y}(vT) -T( \frac{\partial}{\partial y}v + \frac{\partial}{\partial x}u) = \alpha (\frac{\partial^2 }{\partial y^2} \ T ) $$ 


Now we consider continuity equation

$$\frac{\partial}{\partial y}v + \frac{\partial}{\partial x}u = 0$$

And finally get...

$$ \frac{\partial}{\partial x}(uT)   + \frac{\partial}{\partial y}(vT)  = \alpha (\frac{\partial^2 }{\partial y^2} \ T ) $$ 


Now that we have these equations we integrate across y in the BL.

$$ \int_0^Y \frac{\partial}{\partial x}(uT)  dy  + \int_0^Y \frac{\partial}{\partial y}(vT) dy = \int_0^Y \alpha (\frac{\partial^2 }{\partial y^2} \ T ) dy $$ 

$$ \int_0^Y \frac{\partial}{\partial x}(uT)  dy  + (vT)|_Y - (vT)|_0 = \alpha \int_0^Y  (\frac{\partial^2 }{\partial y^2} \ T ) dy $$ 

$$ \int_0^Y \frac{\partial}{\partial x}(uT)  dy  + (vT)|_Y - (vT)|_0 = \alpha [\frac{\partial T}{\partial y}|_Y -  \frac{\partial T}{\partial y} |_0] $$ 



Notice there is a partial derivative with x, we will need to use Leibniz's Rule

\begin{verbatim}

https://en.wikipedia.org/wiki/Leibniz_integral_rule
\end{verbatim}

Consider the integral

$$\int_{y1=a(x)}^{y2=b(x)} f(x,y) dy$$

$$\frac{d}{dx} \left( \int_{y1=a(x)}^{y2=b(x)} f(x,y) dy \right)= f(x,y=b(x)) \frac{d}{dx}b(x) - f(x,y=a(x)) \frac{d}{dx} a(x)  +  \int_{y1=a(x)}^{y2=b(x)} \frac{\partial}{\partial x} f(x,y) dy $$


We can apply it, we let f(x,y) =$uT$

$$\frac{d}{dx} \left( \int_{y1=a(x)}^{y2=b(x)} uT dy \right)= uT(x,y=b(x)) \frac{d}{dx}b(x) - uT(x,y=a(x)) \frac{d}{dx} a(x)  +  \int_{y1=a(x)}^{y2=b(x)} \frac{\partial}{\partial x} uT dy $$

substitute in y1=0, y2=Y

$$\frac{d}{dx} \left( \int_{0}^{y2=Y} uT dy \right)= uT(x,y=Y) \frac{d}{dx}Y(x) - uT(x,y=0) \frac{d}{dx} 0  +  \int_{0}^{y2=Y} \frac{\partial}{\partial x} uT dy $$


$$\frac{d}{dx} \left( \int_{0}^{Y} uT dy \right)= uT(x,y=Y) \frac{d}{dx}Y(x) +  \int_{0}^{Y} \frac{\partial}{\partial x} uT dy $$

$$ \int_{0}^{Y} \frac{\partial}{\partial x} uT dy = \frac{d}{dx} \left( \int_{0}^{Y} uT dy \right) - uT(x,y=Y) \frac{d}{dx}Y(x)$$

substitute back into main equation:

$$ \int_0^Y \frac{\partial}{\partial x}(uT)  dy  + (vT)|_Y - (vT)|_0 = \alpha [\frac{\partial T}{\partial y}|_Y -  \frac{\partial T}{\partial y} |_0] $$ 

$$ \frac{d}{dx} \left( \int_{0}^{Y} uT dy \right) - uT(x,y=Y) \frac{d}{dx}Y(x)  + (vT)|_Y - (vT)|_0 = \alpha [\frac{\partial T}{\partial y}|_Y -  \frac{\partial T}{\partial y} |_0] $$ 


Then consider getting v (y velocity) by integrating continuity equation (\cite{bejan2013convection}).

$$\frac{\partial u}{\partial x} + \frac{\partial v}{\partial y} = 0$$

$$ \int_0^Y \frac{\partial u}{\partial x} dy + \int_0^Y \frac{\partial v}{\partial y} dy = 0$$

$$ \int_0^Y \frac{\partial u}{\partial x} dy + v|_Y - v|_0 = 0$$

$$v|_Y = v|_0 - \int_0^Y \frac{\partial u}{\partial x} dy $$

We considered Leibniz's rule before using f=uT,

$$ \int_{0}^{Y} \frac{\partial}{\partial x} uT dy = \frac{d}{dx} \left( \int_{0}^{Y} uT dy \right) - uT(x,y=Y) \frac{d}{dx}Y(x)$$

now consider f(x,y)=u(x,y)

$$ \int_{0}^{Y} \frac{\partial}{\partial x} u dy = \frac{d}{dx} \left( \int_{0}^{Y} u dy \right) - u(x,y=Y) \frac{d}{dx}Y(x)$$

$$v|_Y = v|_0 - (\frac{d}{dx} \left( \int_{0}^{Y} u dy \right) - u(x,y=Y) \frac{d}{dx}Y(x))$$

$$v|_Y = v|_0 -\frac{d}{dx} \left( \int_{0}^{Y} u dy \right) + u(x,y=Y) \frac{d}{dx}Y(x)$$

Let's substitute this back into our energy equation:

$$ \frac{d}{dx} \left( \int_{0}^{Y} uT dy \right) - uT(x,y=Y) \frac{d}{dx}Y(x)  + (vT)|_Y - (vT)|_0 = \alpha [\frac{\partial T}{\partial y}|_Y -  \frac{\partial T}{\partial y} |_0] $$ 

$$ \frac{d}{dx} \left( \int_{0}^{Y} uT dy \right) - uT(x,y=Y) \frac{d}{dx}Y(x)  + v|_Y T|_Y - (vT)|_0 = \alpha [\frac{\partial T}{\partial y}|_Y -  \frac{\partial T}{\partial y} |_0] $$ 



$$ \frac{d}{dx} \left( \int_{0}^{Y} uT dy \right) - uT(x,y=Y) \frac{d}{dx}Y(x)  + (v|_0 -\frac{d}{dx} \left( \int_{0}^{Y} u dy \right) + u(x,y=Y) \frac{d}{dx}Y(x)) T|_Y - (vT)|_0 $$

$$= \alpha [\frac{\partial T}{\partial y}|_Y -  \frac{\partial T}{\partial y} |_0] $$ 


$$ \frac{d}{dx} \left( \int_{0}^{Y} uT dy \right) - uT(x,y=Y) \frac{d}{dx}Y(x)  + v|_0 T|_Y -\frac{d}{dx} \left( \int_{0}^{Y} u dy \right)T|_Y + u(x,y=Y)T|_Y \frac{d}{dx}Y(x)  - (vT)|_0 $$

$$= \alpha [\frac{\partial T}{\partial y}|_Y -  \frac{\partial T}{\partial y} |_0] $$ 

$$ \frac{d}{dx} \left( \int_{0}^{Y} uT dy \right)   + v|_0 T|_Y -\frac{d}{dx} \left( \int_{0}^{Y} u dy \right)T|_Y  - (vT)|_0 = \alpha [\frac{\partial T}{\partial y}|_Y -  \frac{\partial T}{\partial y} |_0] $$ 

Now we need to introduce some BCs to tidy things up:

\begin{itemize}
\item no slip
\item $Y=\delta_t$
\item conduction at surface $>>$ conduction at BL
\end{itemize}

Firstly no slip v=0 at y=0:


$$ \frac{d}{dx} \left( \int_{0}^{Y} uT dy \right)   -\frac{d}{dx} \left( \int_{0}^{Y} u dy \right)T|_Y   = \alpha [\frac{\partial T}{\partial y}|_Y -  \frac{\partial T}{\partial y} |_0] $$ 

now, Y $=\delta_t (x)$

$$ \frac{d}{dx} \left( \int_{0}^{\delta_t} uT dy \right)   -\frac{d}{dx} \left( \int_{0}^{\delta_t} u dy \right)T|_{\delta_t}   = \alpha [\frac{\partial T}{\partial y}|_{\delta_t} -  \frac{\partial T}{\partial y} |_0] $$ 

note: $T_\infty = T|_{\delta_t}$

$$ \frac{d}{dx} \left( \int_{0}^{\delta_t} uT dy \right)   -\frac{d}{dx} \left( \int_{0}^{\delta_t} u dy \right)T_\infty   = \alpha [\frac{\partial T}{\partial y}|_{\delta_t} -  \frac{\partial T}{\partial y} |_0] $$ 

in general, $T_\infty = T_\infty (x)$

$$ \frac{d}{dx} \left( \int_{0}^{\delta_t} uT dy \right)   - T_\infty (x) \frac{d}{dx} \left( \int_{0}^{\delta_t} u dy \right)    = \alpha [\frac{\partial T}{\partial y}|_{\delta_t} -  \frac{\partial T}{\partial y} |_0] $$ 

Apply product rule:

$$\frac{d}{dx} \left( T_\infty (x) \int_{0}^{\delta_t} u dy \right) = T_\infty (x) \frac{d}{dx} \left( \int_{0}^{\delta_t} u dy \right) + \left( \frac{d T_\infty (x)}{dx} \right) \int_{0}^{\delta_t} u dy  $$

multiply all by -1, im going to just use $T_\infty$ instead of $T_\infty (x)$

$$\frac{d}{dx} \left( T_\infty  \int_{0}^{\delta_t} u dy \right) = T_\infty  \frac{d}{dx} \left( \int_{0}^{\delta_t} u dy \right) + \left( \frac{d T_\infty }{dx} \right) \int_{0}^{\delta_t} u dy  $$

$$-\frac{d}{dx} \left( T_\infty  \int_{0}^{\delta_t} u dy \right) = - T_\infty  \frac{d}{dx} \left( \int_{0}^{\delta_t} u dy \right) - \left( \frac{d T_\infty }{dx} \right) \int_{0}^{\delta_t} u dy  $$

$$-\frac{d}{dx} \left( T_\infty  \int_{0}^{\delta_t} u dy \right) + \left( \frac{d T_\infty }{dx} \right) \int_{0}^{\delta_t} u dy = - T_\infty  \frac{d}{dx} \left( \int_{0}^{\delta_t} u dy \right)   $$

$$ - T_\infty  \frac{d}{dx} \left( \int_{0}^{\delta_t} u dy \right) = -\frac{d}{dx} \left( T_\infty  \int_{0}^{\delta_t} u dy \right) + \left( \frac{d T_\infty }{dx} \right) \int_{0}^{\delta_t} u dy   $$

Substitute back:

$$ \frac{d}{dx} \left( \int_{0}^{\delta_t} uT dy \right)   - T_\infty (x) \frac{d}{dx} \left( \int_{0}^{\delta_t} u dy \right)    = \alpha [\frac{\partial T}{\partial y}|_{\delta_t} -  \frac{\partial T}{\partial y} |_0] $$ 


$$ \frac{d}{dx} \left( \int_{0}^{\delta_t} uT dy \right)   -\frac{d}{dx} \left( T_\infty  \int_{0}^{\delta_t} u dy \right) + \left( \frac{d T_\infty }{dx} \right) \int_{0}^{\delta_t} u dy     = \alpha [\frac{\partial T}{\partial y}|_{\delta_t} -  \frac{\partial T}{\partial y} |_0] $$ 

bring $T_\infty$ into integral

$$ \frac{d}{dx} \left( \int_{0}^{\delta_t} uT dy \right)   -\frac{d}{dx} \left( \int_{0}^{\delta_t} u  T_\infty  dy \right) + \left( \frac{d T_\infty }{dx} \right) \int_{0}^{\delta_t} u dy     = \alpha [\frac{\partial T}{\partial y}|_{\delta_t} -  \frac{\partial T}{\partial y} |_0] $$ 

$$ \frac{d}{dx} \left( \int_{0}^{\delta_t} u(T-T_\infty) dy \right)   + \left( \frac{d T_\infty }{dx} \right) \int_{0}^{\delta_t} u dy     = \alpha [\frac{\partial T}{\partial y}|_{\delta_t} -  \frac{\partial T}{\partial y} |_0] $$ 

on RHS we assume heat conduction at BL negligible compared to heat conduction at surface

$$[\frac{\partial T}{\partial y}|_{\delta_t} -  \frac{\partial T}{\partial y} |_0] \approx - \frac{\partial T}{\partial y} |_0$$

So we are left with our integral BL equation:

$$ \frac{d}{dx} \left( \int_{0}^{\delta_t} u(T-T_\infty) dy \right)   + \left( \frac{d T_\infty }{dx} \right) \int_{0}^{\delta_t} u dy = - \alpha \frac{\partial T}{\partial y} |_0 $$ 

note: $\alpha = \frac{k}{\rho c_p}$

Now assume a power series (cubic) expression for temperature and velocity...

Recall for velocity:

\begin{equation}
\frac{u}{u_\infty} = \frac{3}{2} \frac{y}{\delta_p} - \frac{1}{2} (\frac{y}{\delta_p})^3
\end{equation}

with the BCs:

$$u=0 \ at \  y=0$$
$$u=u_\infty \ at \  y=\delta_p$$
$$\frac{\partial u}{\partial y}= 0 \ at \ y=\delta_p$$
$$\frac{\partial ^2 u}{\partial y^2} = 0 \ at \ y=0$$

and assuming

$$u = a + by + cy^2 + dy^3$$

Near the wall, wall shear stress is constant.

Now we apply the same thing for temperature:

$$T-T_s = a + by + cy^2 + dy^3$$

with the BCs:

$$T-T_s = 0  \ at \ y = 0$$
$$T-T_s= T_\infty -T_s \ at \ y = \delta_t$$

$$\frac{\partial (T-T_s)}{\partial y} = 0 \ at \ y=\delta_t$$
$$\frac{\partial^2 (T-T_s)}{\partial y^2} = 0 \ at \ y=0$$

Last BC is saying, near the wall, heat flux is constant..

Now if we do the same thing for temperature profile, you will get:

$$\frac{T-T_s}{T_\infty-T_s} = \frac{3}{2} \frac{y}{\delta_t} - \frac{1}{2} (\frac{y}{\delta_t})^3$$

$$T = T_s + (T_\infty -T_s)(\frac{3}{2} \frac{y}{\delta_t} - \frac{1}{2} (\frac{y}{\delta_t})^3) $$

we will then need to consider ratio of thermal BL and momentum BL.

Let's try substituting these back into our integral BL equation.

So in case we need:
We already know that from hydrodynamic BL integral and similarity solution

$$\frac{\delta_p}{x} = \frac{4.64}{\sqrt{Re_x}}$$
$$\frac{\delta_p}{x} = \frac{5}{\sqrt{Re_x}}$$

Let's substitute back.

$$ \frac{d}{dx} \left( \int_{0}^{\delta_t} u(T-T_\infty) dy \right)   + \left( \frac{d T_\infty }{dx} \right) \int_{0}^{\delta_t} u dy = - \alpha \frac{\partial T}{\partial y} |_0 $$ 

Assume $T_\infty$ is constant w.r.t x
$$ \frac{d}{dx} \left( \int_{0}^{\delta_t} u(T-T_\infty) dy \right)  = - \alpha \frac{\partial T}{\partial y} |_0 $$ 

Subs the Temp and Velocity profiles:

$$ \frac{d}{dx} \left( \int_{0}^{\delta_t} u_\infty(\frac{3}{2} \frac{y}{\delta_p} - \frac{1}{2} (\frac{y}{\delta_p})^3)(T-T_\infty) dy \right)  = - \alpha \frac{\partial T}{\partial y} |_0 $$ 


$$ \frac{d}{dx} \left( \int_{0}^{\delta_t} u_\infty(\frac{3}{2} \frac{y}{\delta_p} - \frac{1}{2} (\frac{y}{\delta_p})^3)(T-T_\infty) dy \right)  = - \alpha \frac{\partial T}{\partial y} |_0 $$ 

subs:
$$T = T_s + (T_\infty -T_s)(\frac{3}{2} \frac{y}{\delta_t} - \frac{1}{2} (\frac{y}{\delta_t})^3) $$

$$ \frac{d}{dx} \left( \int_{0}^{\delta_t} u_\infty(\frac{3}{2} \frac{y}{\delta_p} - \frac{1}{2} (\frac{y}{\delta_p})^3)((T_s + (T_\infty -T_s)(\frac{3}{2} \frac{y}{\delta_t} - \frac{1}{2} (\frac{y}{\delta_t})^3))-T_\infty) dy \right)  = - \alpha \frac{\partial T}{\partial y} |_0 $$ 

$$ \frac{d}{dx} \left( \int_{0}^{\delta_t} u_\infty(\frac{3}{2} \frac{y}{\delta_p} - \frac{1}{2} (\frac{y}{\delta_p})^3)((T_s-T_\infty + (T_\infty -T_s)(\frac{3}{2} \frac{y}{\delta_t} - \frac{1}{2} (\frac{y}{\delta_t})^3))) dy \right)  = - \alpha \frac{\partial T}{\partial y} |_0 $$ 

$$ \frac{d}{dx} \left( \int_{0}^{\delta_t} u_\infty(\frac{3}{2} \frac{y}{\delta_p} - \frac{1}{2} (\frac{y}{\delta_p})^3)(T_s-T_\infty - (T_s-T_\infty)(\frac{3}{2} \frac{y}{\delta_t} - \frac{1}{2} (\frac{y}{\delta_t})^3)) dy \right)  = - \alpha \frac{\partial T}{\partial y} |_0 $$ 

$$ \frac{d}{dx} \left( \int_{0}^{\delta_t} u_\infty(T_s-T_\infty )(\frac{3}{2} \frac{y}{\delta_p} - \frac{1}{2} (\frac{y}{\delta_p})^3)(1-(\frac{3}{2} \frac{y}{\delta_t} - \frac{1}{2} (\frac{y}{\delta_t})^3)) dy \right)  = - \alpha \frac{\partial T}{\partial y} |_0 $$ 

$$ u_\infty(T_s-T_\infty ) \frac{d}{dx} \left( \int_{0}^{\delta_t} (\frac{3}{2} \frac{y}{\delta_p} - \frac{1}{2} (\frac{y}{\delta_p})^3)(1-(\frac{3}{2} \frac{y}{\delta_t} - \frac{1}{2} (\frac{y}{\delta_t})^3)) dy \right)  = - \alpha \frac{\partial T}{\partial y} |_0 $$ 

divide both sides by $(T_s-T_\infty)$, 

$$ u_\infty \frac{d}{dx} \left( \int_{0}^{\delta_t} (\frac{3}{2} \frac{y}{\delta_p} - \frac{1}{2} (\frac{y}{\delta_p})^3)(1-(\frac{3}{2} \frac{y}{\delta_t} - \frac{1}{2} (\frac{y}{\delta_t})^3)) dy \right)  =  -\alpha (\frac{1}{(T_s-T_\infty )}) \frac{\partial T-T_s}{\partial y} |_0 $$ 

$$ u_\infty \frac{d}{dx} \left( \int_{0}^{\delta_t} (\frac{3}{2} \frac{y}{\delta_p} - \frac{1}{2} (\frac{y}{\delta_p})^3)(1-(\frac{3}{2} \frac{y}{\delta_t} - \frac{1}{2} (\frac{y}{\delta_t})^3)) dy \right)  =  \alpha (\frac{1}{(T_\infty- T_s )}) \frac{\partial T-T_s}{\partial y} |_0 $$ 



$$ u_\infty \frac{d}{dx} \left( \int_{0}^{\delta_t} (\frac{3}{2} \frac{y}{\delta_p} - \frac{1}{2} (\frac{y}{\delta_p})^3)(1-(\frac{3}{2} \frac{y}{\delta_t} - \frac{1}{2} (\frac{y}{\delta_t})^3)) dy \right)  = \alpha \frac{\partial \theta}{\partial y} |_0 $$ 

Where $\theta= \frac{T-T_s}{T_\infty-T_s}$

Let's evaluate the integral:

$$\int_{0}^{\delta_t} (\frac{3}{2} \frac{y}{\delta_p} - \frac{1}{2} (\frac{y}{\delta_p})^3)(1-(\frac{3}{2} \frac{y}{\delta_t} - \frac{1}{2} (\frac{y}{\delta_t})^3)) dy$$

let's substitute the integral for dy with $d \frac{y}{\delta_t}$

$$dy = \delta_t d \frac{y}{\delta_t}$$

note: $\delta_t$ and $\delta_p$ are constant w.r.t y

subs the change of variable:


$$\int_{0}^{\delta_t} (\frac{3}{2} \frac{y}{\delta_p} - \frac{1}{2} (\frac{y}{\delta_p})^3)(1-(\frac{3}{2} \frac{y}{\delta_t} - \frac{1}{2} (\frac{y}{\delta_t})^3)) dy$$

$$=\int_{0}^{\delta_t} (\frac{3}{2} \frac{y}{\delta_p} - \frac{1}{2} (\frac{y}{\delta_p})^3)(1-(\frac{3}{2} \frac{y}{\delta_t} - \frac{1}{2} (\frac{y}{\delta_t})^3)) d \frac{y}{\delta_t} \delta_t $$

$$= \delta_t  \int_{0}^{1} (\frac{3}{2} \frac{y}{\delta_p} - \frac{1}{2} (\frac{y}{\delta_p})^3)(1-(\frac{3}{2} \frac{y}{\delta_t} - \frac{1}{2} (\frac{y}{\delta_t})^3)) d \frac{y}{\delta_t} $$

let's replace $\beta = \frac{y}{\delta_t}$, $y=\beta \delta_t$

$$= \delta_t  \int_{0}^{1} (\frac{3}{2} \frac{y}{\delta_p} - \frac{1}{2} (\frac{y}{\delta_p})^3)(1-(\frac{3}{2} \beta- \frac{1}{2} \beta)^3) d \beta $$


$$= \delta_t  \int_{0}^{1} (\frac{3}{2} \frac{y}{\delta_p} - \frac{1}{2} (\frac{y}{\delta_p})^3)(1-\frac{3}{2} \beta + \frac{1}{2} \beta^3) d \beta $$

$$= \delta_t  \int_{0}^{1} (\frac{3}{2} \frac{\beta \delta_t}{\delta_p} - \frac{1}{2} (\frac{\beta \delta_t}{\delta_p})^3)(1-\frac{3}{2} \beta + \frac{1}{2} \beta^3) d \beta $$

$$= \delta_t  \int_{0}^{1} \frac{3}{2} \frac{\beta \delta_t}{\delta_p} (1-\frac{3}{2} \beta + \frac{1}{2} \beta^3) - \frac{1}{2} (\frac{\beta \delta_t}{\delta_p})^3(1-\frac{3}{2} \beta + \frac{1}{2} \beta^3) d \beta $$

$$= \delta_t  \int_{0}^{1} \frac{3}{2} \frac{\beta \delta_t}{\delta_p} (1-\frac{3}{2} \beta + \frac{1}{2} \beta^3) d \beta  - \delta_t  \int_{0}^{1} \frac{1}{2} (\frac{\beta \delta_t}{\delta_p})^3(1-\frac{3}{2} \beta + \frac{1}{2} \beta^3) d \beta $$

[careless mistake! multiplied in $\beta$ not $\beta^3$]

$$= \delta_t \frac{\delta_t}{\delta_p} \int_{0}^{1} \frac{3}{2} \beta  (1-\frac{3}{2} \beta + \frac{1}{2} \beta^3) d \beta  - \delta_t (\frac{ \delta_t}{\delta_p})^3 \int_{0}^{1} \frac{1}{2}\beta^3 (1-\frac{3}{2} \beta + \frac{1}{2} \beta^3) d \beta $$

$$= \delta_t \frac{\delta_t}{\delta_p} \int_{0}^{1} \frac{3}{2}   (\beta-\frac{3}{2} \beta^2 + \frac{1}{2} \beta^4) d \beta  - \delta_t (\frac{ \delta_t}{\delta_p})^3 \int_{0}^{1} \frac{1}{2} (\beta^3-\frac{3}{2} \beta^4 + \frac{1}{2} \beta^6) d \beta $$

$$=  \frac{3}{2} \delta_t \frac{\delta_t}{\delta_p}  \int_{0}^{1}    (\beta-\frac{3}{2} \beta^2 + \frac{1}{2} \beta^4) d \beta  - \frac{1}{2} \delta_t (\frac{ \delta_t}{\delta_p})^3 \int_{0}^{1}  (\beta^3-\frac{3}{2} \beta^4 + \frac{1}{2} \beta^6) d \beta $$

$$=  \frac{3}{2} \delta_t \frac{\delta_t}{\delta_p}  \int_{0}^{1}    (\beta-\frac{3}{2} \beta^2 + \frac{1}{2} \beta^4) d \beta  - \frac{1}{2} \delta_t (\frac{ \delta_t}{\delta_p})^3 \int_{0}^{1}  (\beta^3-\frac{3}{2} \beta^4 + \frac{1}{2} \beta^6) d \beta $$

$$=  \frac{3}{2} \delta_t \frac{\delta_t}{\delta_p}  \frac{1}{10}  - \frac{1}{2} \delta_t (\frac{ \delta_t}{\delta_p})^3 \frac{3}{140} $$


$$=  \frac{3}{20} \delta_t \frac{\delta_t}{\delta_p}   - \frac{3}{280} \delta_t (\frac{ \delta_t}{\delta_p})^3 $$

Substituting the integral back:

$$ u_\infty \frac{d}{dx} \left(  \frac{3}{20} \delta_t \frac{\delta_t}{\delta_p}   - \frac{3}{280} \delta_t (\frac{ \delta_t}{\delta_p})^3 \right)  = \alpha \frac{\partial \theta}{\partial y} |_0 $$ 

$$ u_\infty \frac{d}{dx} \left(  \frac{3}{20} \frac{\delta_t^2}{\delta_p}   - \frac{3}{280} (\frac{ \delta_t^4}{\delta_p^3}) \right)  = \alpha \frac{\partial \theta}{\partial y} |_0 $$ 

in approximate method, we consider the ratio $\frac{\delta_t}{\delta_p}$


if $\frac{\delta_t}{\delta_p} >> 1$ we can ignore first term, otherwise, if $\frac{\delta_t}{\delta_p} << 1$ we can ignore the second term

$$\frac{\delta_t}{\delta_p} >> 1$$

$$ u_\infty \frac{d}{dx} \left(   - \frac{3}{280} (\frac{ \delta_t^4}{\delta_p^3}) \right)  = \alpha \frac{\partial \theta}{\partial y} |_0 $$ 

$$\frac{\delta_t}{\delta_p} << 1$$

$$ u_\infty \frac{d}{dx} \left(  \frac{3}{20} \frac{\delta_t^2}{\delta_p}   \right)  = \alpha \frac{\partial \theta}{\partial y} |_0 $$ 

Now we can substitute $\theta$

$$\theta = \frac{T-T_s}{T_\infty-T_s} = \frac{3}{2} \frac{y}{\delta_t} - \frac{1}{2} (\frac{y}{\delta_t})^3$$

$$ \frac{\partial }{\partial y}\theta = \frac{3}{2} \frac{1}{\delta_t} - \frac{1}{2} (\frac{3y^2}{\delta_t^3})$$
now set y=0

$$\frac{\partial }{\partial y}\theta |_{y=0} = \frac{3}{2\delta_t}$$

subs back in:

$$\frac{\delta_t}{\delta_p} >> 1$$

$$ u_\infty \frac{d}{dx} \left(   - \frac{3}{280} (\frac{ \delta_t^4}{\delta_p^3}) \right)  = \alpha \frac{3}{2\delta_t} $$ 

$$\frac{\delta_t}{\delta_p} << 1$$

$$ u_\infty \frac{d}{dx} \left(  \frac{3}{20} \frac{\delta_t^2}{\delta_p}   \right)  = \alpha \frac{3}{2\delta_t}$$ 

note: from momentum integral equation

$$\delta_p = x \frac{4.64}{\sqrt{Re_x}} = 4.64\sqrt{\frac{\nu x}{u_\infty}}$$

Try substituting in latter case 

$$\frac{\delta_t}{\delta_p} << 1$$
$$ u_\infty \frac{d}{dx} \left(  \frac{3}{20} \frac{\delta_t^2}{\delta_p}   \right)  = \alpha \frac{3}{2\delta_t}$$ 

$$ u_\infty \frac{d}{dx} \left(  \frac{3}{20} \frac{\delta_t^2}{ 4.64\sqrt{\frac{\nu x}{u_\infty}}}   \right)  = \alpha \frac{3}{2\delta_t}$$ 


$$  \frac{u_\infty^{1.5}}{\sqrt{\nu}} \frac{3}{20*4.64} \frac{d}{dx} \left(   \frac{\delta_t^2}{\sqrt{x}}   \right)  = \alpha \frac{3}{2\delta_t}$$ 

using quotient rule [careless mistake here, forgot to multiply by 2]:

$$\frac{d}{dx} \left(   \frac{\delta_t^2}{\sqrt{x}}   \right) = \frac{2\sqrt{x} \delta_t \frac{d \delta_t}{d x} - \delta_t^2 (0.5) \frac{1}{\sqrt{x}}}{x}$$

substitute back:

$$  \frac{u_\infty^{1.5}}{\sqrt{\nu}} \frac{3}{20*4.64} \frac{2 \sqrt{x} \delta_t \frac{d \delta_t}{d x} - \delta_t^2 (0.5) \frac{1}{\sqrt{x}}}{x}  = \alpha \frac{3}{2\delta_t}$$ 

$$  \frac{u_\infty^{1.5}}{\sqrt{\nu}} \frac{3}{20*4.64} (\frac{2}{ \sqrt{x}} \delta_t \frac{d \delta_t}{d x} - \delta_t^2 (0.5) \frac{1}{x\sqrt{x}})  = \alpha \frac{3}{2\delta_t}$$ 

$$  \frac{u_\infty^{1.5}}{\sqrt{\nu}} \frac{3}{20*4.64} (\frac{2}{\sqrt{x}} \delta_t^2 \frac{d \delta_t}{d x} - \delta_t^3 (0.5) \frac{1}{x\sqrt{x}})  = \alpha \frac{3}{2}$$ 


$$\frac{d \delta_t^3}{dx} = 3 \delta_t^2 \frac{d \delta_t}{dx}$$


$$ \frac{1}{3} \frac{d \delta_t^3}{dx} =  \delta_t^2 \frac{d \delta_t}{dx}$$


substitute back:

$$  \frac{u_\infty^{1.5}}{\sqrt{\nu}} \frac{3}{20*4.64} (\frac{1}{\sqrt{x}}\frac{2}{3} \frac{d \delta_t^3}{dx} - \delta_t^3 (0.5) \frac{1}{x\sqrt{x}})  = \alpha \frac{3}{2}$$ 

1st order linear ODE in $\delta_t^3$

$$  \frac{u_\infty^{1.5}}{\sqrt{\nu}} (\frac{1}{\sqrt{x}}\frac{2}{3} \frac{d \delta_t^3}{dx} - \delta_t^3 (0.5) \frac{1}{x\sqrt{x}})  =  \alpha \frac{20*4.64}{3}  \frac{3}{2}$$ 

$$  \frac{1}{\sqrt{x}}\frac{2}{3} \frac{d \delta_t^3}{dx} - \delta_t^3 (0.5) \frac{1}{x\sqrt{x}}  =  \alpha \frac{10*4.64}{1} \frac{\sqrt{\nu}}{u_\infty^{1.5}} $$ 

$$  \frac{d \delta_t^3}{dx} - \delta_t^3 (0.75) \frac{1}{x}  =  \alpha \frac{15*4.64}{1} \frac{\sqrt{\nu}}{u_\infty^{1.5}} \sqrt{x} $$ 

Let 

$$A = \alpha \frac{15*4.64}{1} \frac{\sqrt{\nu}}{u_\infty^{1.5}} $$

$$  \frac{d \delta_t^3}{dx} - \delta_t^3  \frac{0.75}{x}  =  A \sqrt{x} $$ 




[Alternate solution: substitute later, may get better solution (less careless mistake)]

$$ u_\infty \frac{d}{dx} \left(  \frac{3}{20} \frac{\delta_t^2}{\delta_p}   \right)  = \alpha \frac{3}{2\delta_t}$$ 

$$ u_\infty \frac{d}{dx} \left(  \frac{3}{20} \delta_p \frac{\delta_t^2}{\delta_p^2}   \right)  = \alpha \frac{3}{2\delta_t}$$ 

$$ u_\infty \frac{d}{dx} \left(  \frac{3}{20} \delta_p \frac{\delta_t^2}{\delta_p^2}   \right)  = \alpha \frac{3}{2\delta_t}$$ 

let $\phi= \frac{\delta_t}{\delta_p}$

$$ u_\infty \frac{d}{dx} \left(  \frac{3}{20} \delta_p \phi^2 \right)  = \alpha \frac{3}{2\delta_t}$$ 

$$ u_\infty \frac{d}{dx} \left(  \delta_p \phi^2 \right)  = \alpha \frac{10}{\delta_t}$$ 


Product rule:

$$ \phi^2 \frac{d}{dx} \delta_p  + \delta_p \frac{d}{dx} \phi^2  = \alpha \frac{10}{  u_\infty \delta_t}$$ 

$$ \delta_t \phi^2 \frac{d}{dx} \delta_p  + \delta_t \delta_p \frac{d}{dx} \phi^2  = \alpha \frac{10}{  u_\infty }$$ 

sub $\delta_t = \phi \delta_p$

$$ \delta_p \phi^3 \frac{d}{dx} \delta_p  + \phi \delta_p^2 \frac{d}{dx} \phi^2  = \alpha \frac{10}{  u_\infty }$$ 

$$ \delta_p \phi^3 \frac{d}{dx} \delta_p  + 2 \phi^2 \delta_p^2 \frac{d}{dx} \phi  = \alpha \frac{10}{  u_\infty }$$ 


$$  \phi^3 \delta_p \frac{d}{dx} \delta_p  + 2 \delta_p^2  \phi^2 \frac{d}{dx} \phi^2  = \alpha \frac{10}{  u_\infty }$$ 

$$  \phi^3 \delta_p \frac{d}{dx} \delta_p  +  \delta_p^2  \frac{2}{3} \frac{d}{dx} \phi^3  = \alpha \frac{10}{  u_\infty }$$ 

$$\delta_p = x \frac{4.64}{\sqrt{Re_x}} = \frac{4.64 \sqrt{\nu x}}{\sqrt{u_\infty}}$$

$$\delta_p = x \frac{4.64}{\sqrt{Re_x}} = \frac{4.64 \sqrt{\nu x}}{\sqrt{u_\infty}}$$

$$\delta_p^2 = x \frac{4.64^2}{\sqrt{Re_x}} = \frac{4.64^2 \nu x}{u_\infty}$$


$$\frac{d}{dx} \delta_p = \frac{1}{2} \frac{4.64 \sqrt{\nu}}{\sqrt{u_\infty x}}$$

$$\delta_p \frac{d}{dx} \delta_p = \frac{1}{2}  \frac{4.64 \sqrt{\nu x}}{\sqrt{u_\infty}} \frac{4.64 \sqrt{\nu}}{\sqrt{u_\infty x}}$$

$$\delta_p \frac{d}{dx} \delta_p = \frac{1}{2}  \frac{4.64^2 \nu}{u_\infty} $$

subs back:

$$  \phi^3 \delta_p \frac{d}{dx} \delta_p  +  \delta_p^2  \frac{2}{3} \frac{d}{dx} \phi^3  = \alpha \frac{10}{  u_\infty }$$ 

$$  \phi^3 \frac{1}{2}  \frac{4.64^2 \nu}{u_\infty}  + \frac{4.64^2 \nu x}{u_\infty} \frac{2}{3} \frac{d}{dx} \phi^3  = \alpha \frac{10}{  u_\infty }$$ 


$$  \phi^3 \frac{1}{2}  \frac{4.64^2 \nu}{1}  + \frac{4.64^2 \nu x}{1} \frac{2}{3} \frac{d}{dx} \phi^3  = \alpha \frac{10}{ 1 }$$ 

$$  \phi^3 \frac{1}{2}   + x \frac{2}{3} \frac{d}{dx} \phi^3  = \frac{ 10}{4.64^2} \frac{\alpha}{ \nu  }$$ 

$$  \phi^3 \frac{3}{2}   +2 x \frac{d}{dx} \phi^3  = \frac{30}{4.64^2} \frac{1}{Pr }$$ 
$$  \phi^3    + \frac{4}{3} x \frac{d}{dx} \phi^3  = \frac{20}{4.64^2} \frac{1}{Pr }$$ 



First order ODE:

\begin{verbatim}
https://mathworld.wolfram.com/First-OrderOrdinaryDifferentialEquation.html
\end{verbatim}
$$y' + p(x) y = q(x)$$


$$y = \frac{\int e^{\int^x p(x') dx'} q(x) dx + c }{e^{\int^x p(x') dx'}}$$

Replace x by dummy variables x'

$$  \frac{d \delta_t^3}{dx'} - \delta_t^3  \frac{0.75}{x'}  =  A \sqrt{x'} $$ 

$$p(x') = -\frac{0.75}{x'}$$

$$q(x') = A \sqrt{x'}$$

[second careless mistake: wrong choice of integrating factor]
$$ \int^x p(x') dx' = \int^x \frac{-0.75}{x'} dx' = -0.75 \ln (x)  $$ 


$$\exp (0.75 \ln (x)) = -\exp(0.75) x$$


$$\int e^{\int^x p(x') dx'} q(x) dx  = -\int \exp(0.75) x  A \sqrt{x} dx  $$

$$=-A \exp (0.75) \int x^{1.5} dx $$
$$=-A \exp (0.75) \frac{x^{2.5}}{2.5}$$

$$\delta_t^3 = \frac{-A \exp (0.75) \frac{x^{2.5}}{2.5} + c }{-\exp(0.75) x}$$

$$\delta_t^3 = \frac{-A \exp (0.75) \frac{x^{2.5}}{2.5} }{-\exp(0.75) x}+\frac{c }{\exp(0.75) x}$$

$$\delta_t^3 = A  \frac{x^{1.5}}{2.5}  +\frac{c }{-\exp(0.75) x}$$

[wrong... ODE solving gone wrong...]

\begin{verbatim}
https://www.wolframalpha.com/input/?i=solve+y%27+-+0.75+y%2Ft+%3DA+sqrt%28t%29
\end{verbatim}

According to wolfram:

$$y = \frac{4}{3} A x^{1.5} + C x^{0.75}$$

apply BC, x=0 $\delta_t= 0$, c=0

$$\delta_t^3 = \frac{4}{3} A  x^{1.5}  $$

$$\delta_t = \mathcal{O}( \sqrt{x})$$

$$\delta_t^3 =  \alpha \frac{15*4.64}{1} \frac{\sqrt{\nu}}{u_\infty^{1.5}}  x^{1.5} \frac{4}{3}  $$

$$\frac{\delta_p}{x} = \frac{4.64}{\sqrt{Re_x}}$$

$$\delta_t^3 =  \frac{\alpha}{\nu} \frac{15*4.64}{1} \frac{\nu^{1.5}}{u_\infty^{1.5}}   \frac{4}{3} x^{1.5} $$


$$\frac{x}{\sqrt{Re_x}} = \frac{\sqrt{\nu x}}{\sqrt{u_\infty}}$$

$$\delta_t^3 =  \frac{\alpha}{\nu} \frac{15*4.64}{1} \frac{x^3}{Re_x^{1.5}}   \frac{4}{3}   $$

$$\frac{\delta_p}{4.64} = \frac{x}{\sqrt{Re_x}}$$


$$\delta_t^3 =  \frac{1}{Pr} \frac{15*4.64}{1} (\frac{\delta_p}{4.64} )^3  \frac{4}{3} $$


$$\frac{\delta_t^3}{\delta_p^3} = \frac{1}{Pr} \frac{15*4.64}{4.64^3} \frac{4}{3}$$

$$\frac{\delta_t^3}{\delta_p^3} = \frac{1}{Pr} 0.92895$$

$$\frac{\delta_t}{\delta_p} = 0.97573 \frac{1}{Pr^{1/3}}$$


[redundant to use similarity soln and wrong number]
If you use the similarity solution, you will get this:

$$\frac{\delta_p}{x} = \frac{5}{\sqrt{Re_x}}$$


$$\delta_t^3 =  \frac{1}{Pr} \frac{30*5}{1} (\frac{\delta_p}{5} )^3   \frac{1}{2.5}  $$

$$\frac{\delta_t}{\delta_p} = 0.78297 \frac{1}{Pr^{1/3}}$$
[redundant to use similarity soln and wrong number]

What's our heat flux?


$$\frac{\partial}{\partial y} \theta |_{y=0} = \frac{3}{2 \delta_t}$$

$$\frac{\partial}{\partial y} \theta |_{y=0} = \frac{3}{2 \delta_p *0.97573 \frac{1}{Pr^{1/3}}}$$


$$\frac{\partial}{\partial y} \theta |_{y=0} = \frac{3}{2 \delta_p *0.97573} * Pr^{1/3}$$


$$\delta_p = \frac{4.64 x}{\sqrt{Re_x}}$$

$$\frac{\partial}{\partial y} \theta |_{y=0} = \frac{3}{2 \frac{4.64 x}{\sqrt{Re_x}} *0.97573} * Pr^{1/3}$$

$$\frac{\partial}{\partial y} \theta |_{y=0} = \frac{3}{2 *4.64 x *0.97573} * Pr^{1/3} Re^{1/2}$$

$$Nu_x = \frac{h x}{k}$$

$$q''_{wall \rightarrow fluid} = -k \frac{\partial T}{\partial y}=-h(T_\infty-T_s)$$

$$-k \frac{\partial T}{\partial y}=-h(T_\infty-T_s)$$

$$\frac{\partial T}{\partial y} = \frac{h}{k}(T_\infty-T_s)$$

$$\theta = \frac{T-T_s}{T_\infty - T_s}$$

$$\frac{\partial \theta}{\partial y} = \frac{h}{k}$$

$$\frac{h}{k}  |_{y=0} = \frac{3}{2 *4.64 x *0.97573} * Pr^{1/3} Re^{1/2}$$

$$\frac{hx}{k}  |_{y=0} = \frac{3}{2 *4.64 *0.97573} * Pr^{1/3} Re^{1/2}$$

$$Nu_x |_{y=0} = \frac{3}{2 *4.64 *0.97573} * Pr^{1/3} Re^{1/2}$$
$$Nu_x |_{y=0} = 0.3313 Pr^{1/3} Re_x^{1/2}$$
$$Nu_L |_{y=0} = 0.663 Pr^{1/3} Re_L^{1/2}$$


\part{Natural Convection}

\section{Simplifying N-S equations and Energy Equations}

Mass, momentum and energy balances:
$$  \frac{\partial}{\partial x} u +  \frac{\partial}{\partial y} v +  \frac{\partial}{\partial z} w = 0$$


$$\frac{\partial }{\partial t} u + u \frac{\partial}{\partial x} u + v \frac{\partial}{\partial y} u + w \frac{\partial}{\partial z} u - \nu ( \frac{\partial^2}{\partial x^2} u + \frac{\partial^2}{\partial y^2} \ u + \frac{\partial^2}{\partial z^2} u) = - \frac{1}{\rho_0} \frac{\partial P}{\partial x} +g_x$$

$$\frac{\partial }{\partial t} v + u \frac{\partial}{\partial x} v + v \frac{\partial}{\partial y} v + w \frac{\partial}{\partial z} v - \nu ( \frac{\partial^2}{\partial x^2} v + \frac{\partial^2}{\partial y^2} \ v + \frac{\partial^2}{\partial z^2} v) = - \frac{1}{\rho_0} \frac{\partial P}{\partial y} +g_y$$

$$\frac{\partial }{\partial t} w + u \frac{\partial}{\partial x} w + v \frac{\partial}{\partial y} w + w \frac{\partial}{\partial z} w - \nu ( \frac{\partial^2}{\partial x^2} w + \frac{\partial^2}{\partial y^2} \ w + \frac{\partial^2}{\partial z^2} w) = - \frac{1}{\rho_0} \frac{\partial P}{\partial z} +g_z$$


$$\frac{\partial }{\partial t} T + u \frac{\partial}{\partial x} T + v \frac{\partial }{\partial y} \ T + w \frac{\partial }{\partial z} T  = \alpha ( \frac{\partial^2}{\partial x^2} T +  \frac{\partial^2 }{\partial y^2} \ T +  \frac{\partial^2 }{\partial z^2} T) $$ 


Note: compressibility applies here! How can we deal with this?

$$  \frac{\partial}{\partial x} \rho u +  \frac{\partial}{\partial y}  \rho v +  \frac{\partial}{\partial z} \rho w = 0$$

(note: steady state only)

$$u \frac{\partial}{\partial x} \rho  + v \frac{\partial}{\partial y}  \rho  +  w \frac{\partial}{\partial z} \rho   + \rho \left[ \frac{\partial}{\partial x} u +  \frac{\partial}{\partial y} v +  \frac{\partial}{\partial z} w \right] = 0$$

$$u \frac{\partial}{\partial x} \rho  + v \frac{\partial}{\partial y}  \rho  +  w \frac{\partial}{\partial z} \rho   + \rho \left[ \frac{\partial}{\partial x} u +  \frac{\partial}{\partial y} v +  \frac{\partial}{\partial z} w \right] = 0$$

We do term by term analysis:
$$u \frac{\partial }{\partial x} \rho + \rho \frac{\partial u}{\partial x}$$

$$= \mathcal{O}(u_{max} \frac{0.02 \rho}{\delta_t})+ \mathcal{O}(\rho \frac{ u_{max}}{\delta_t})$$

$$u \frac{\partial }{\partial x} \rho + \rho \frac{\partial u}{\partial x} \approx \rho \frac{\partial u}{\partial x}$$

The change in density is considered small in comparison to original fluid density... (Boussinesq approximation)

Therefore,
$$  \frac{\partial}{\partial x} u +  \frac{\partial}{\partial y} v +  \frac{\partial}{\partial z} w = 0$$

With this equation, we can still use essentially the same form of the (mostly) incompressible momentum equations:

$$\frac{\partial }{\partial t} u + u \frac{\partial}{\partial x} u + v \frac{\partial}{\partial y} u + w \frac{\partial}{\partial z} u - \nu ( \frac{\partial^2}{\partial x^2} u + \frac{\partial^2}{\partial y^2} \ u + \frac{\partial^2}{\partial z^2} u) = - \frac{1}{\rho_0} \frac{\partial P}{\partial x} +g_x$$

$$\frac{\partial }{\partial t} v + u \frac{\partial}{\partial x} v + v \frac{\partial}{\partial y} v + w \frac{\partial}{\partial z} v - \nu ( \frac{\partial^2}{\partial x^2} v + \frac{\partial^2}{\partial y^2} \ v + \frac{\partial^2}{\partial z^2} v) = - \frac{1}{\rho_0} \frac{\partial P}{\partial y} +g_y$$

$$\frac{\partial }{\partial t} w + u \frac{\partial}{\partial x} w + v \frac{\partial}{\partial y} w + w \frac{\partial}{\partial z} w - \nu ( \frac{\partial^2}{\partial x^2} w + \frac{\partial^2}{\partial y^2} \ w + \frac{\partial^2}{\partial z^2} w) = - \frac{1}{\rho_0} \frac{\partial P}{\partial z} +g_z$$

For simplicity, we assume a 2D type flow, where convection goes upwards in the positive y axis. This is to say, w=0 everywhere, $g_z=0$

$$0 = - \frac{1}{\rho_0} \frac{\partial P}{\partial z} $$

we can also apply the same for the x and y momentum eqns

$$\frac{\partial }{\partial t} u + u \frac{\partial}{\partial x} u + v \frac{\partial}{\partial y} u + w \frac{\partial}{\partial z} u - \nu ( \frac{\partial^2}{\partial x^2} u + \frac{\partial^2}{\partial y^2} \ u + \frac{\partial^2}{\partial z^2} u) = - \frac{1}{\rho_0} \frac{\partial P}{\partial x} +g_x$$

$$\frac{\partial }{\partial t} v + u \frac{\partial}{\partial x} v + v \frac{\partial}{\partial y} v + w \frac{\partial}{\partial z} v - \nu ( \frac{\partial^2}{\partial x^2} v + \frac{\partial^2}{\partial y^2} \ v + \frac{\partial^2}{\partial z^2} v) = - \frac{1}{\rho_0} \frac{\partial P}{\partial y} +g_y$$

No variation of u and v in z direction
Thus resulting in:

$$\frac{\partial }{\partial t} u + u \frac{\partial}{\partial x} u + v \frac{\partial}{\partial y} u  - \nu ( \frac{\partial^2}{\partial x^2} u + \frac{\partial^2}{\partial y^2} \ u ) = - \frac{1}{\rho_0} \frac{\partial P}{\partial x} +g_x$$

$$\frac{\partial }{\partial t} v + u \frac{\partial}{\partial x} v + v \frac{\partial}{\partial y} v  - \nu ( \frac{\partial^2}{\partial x^2} v + \frac{\partial^2}{\partial y^2} \ v ) = - \frac{1}{\rho_0} \frac{\partial P}{\partial y} +g_y$$

With gravity being only in y direction,

$$\frac{\partial }{\partial t} u + u \frac{\partial}{\partial x} u + v \frac{\partial}{\partial y} u  - \nu ( \frac{\partial^2}{\partial x^2} u + \frac{\partial^2}{\partial y^2} \ u ) = - \frac{1}{\rho_0} \frac{\partial P}{\partial x} $$

$$\frac{\partial }{\partial t} v + u \frac{\partial}{\partial x} v + v \frac{\partial}{\partial y} v  - \nu ( \frac{\partial^2}{\partial x^2} v + \frac{\partial^2}{\partial y^2} \ v ) = - \frac{1}{\rho_0} \frac{\partial P}{\partial y} +g_y$$

With steady state assumption:

$$ u \frac{\partial}{\partial x} u + v \frac{\partial}{\partial y} u  - \nu ( \frac{\partial^2}{\partial x^2} u + \frac{\partial^2}{\partial y^2} \ u ) = - \frac{1}{\rho_0} \frac{\partial P}{\partial x} $$

$$ u \frac{\partial}{\partial x} v + v \frac{\partial}{\partial y} v  - \nu ( \frac{\partial^2}{\partial x^2} v + \frac{\partial^2}{\partial y^2} \ v ) = - \frac{1}{\rho_0} \frac{\partial P}{\partial y} +g_y$$


Thus we have our full set of hydrodynamic equations to begin:


$$  \frac{\partial}{\partial x} u +  \frac{\partial}{\partial y} v  = 0$$

$$ u \frac{\partial}{\partial x} u + v \frac{\partial}{\partial y} u  - \nu ( \frac{\partial^2}{\partial x^2} u + \frac{\partial^2}{\partial y^2} \ u ) = - \frac{1}{\rho_0} \frac{\partial P}{\partial x} $$

$$ u \frac{\partial}{\partial x} v + v \frac{\partial}{\partial y} v  - \nu ( \frac{\partial^2}{\partial x^2} v + \frac{\partial^2}{\partial y^2} \ v ) = - \frac{1}{\rho_0} \frac{\partial P}{\partial y} +g_y$$

%Now what about our energy equations?

$$\frac{\partial }{\partial t} T + u \frac{\partial}{\partial x} T + v \frac{\partial }{\partial y} \ T + w \frac{\partial }{\partial z} T  = \alpha ( \frac{\partial^2}{\partial x^2} T +  \frac{\partial^2 }{\partial y^2} \ T +  \frac{\partial^2 }{\partial z^2} T) $$ 

We make the assumption that the change in density is so small it does not significantly impact thermal diffusivity. Again if we expand things out via product rule, the terms with change in density are assumed small compared to other terms.

$$ u \frac{\partial}{\partial x} T + v \frac{\partial }{\partial y} \ T = \alpha ( \frac{\partial^2}{\partial x^2} T +  \frac{\partial^2 }{\partial y^2} \ T ) $$ 

\section{Boussinesq approximation and dimensionless analysis}
Now where does the buoyancy force come from?

%Think in terms of intuition...

It comes from surrounding fluid of heavier density "pushing" the lighter fluid upwards.

We expect buoyancy force to come from the pressure term. How do we then put this into our equations?

Consider this dimensionless analysis that as long as we are within the boundary layer, (\cite{bejan2013convection}):

$$\mathcal{O}(x) = \delta_T$$
$$\mathcal{O}(y) = H \ (wall\ height)$$
$$H >> \delta_T$$


(why thermal boundary layer and not momentum boundary layer? because the underlying driving force is temperature, that's the more important BL in natural convection)

Also static pressure should change only with y, not x.

What's the expression then?

$$P_\infty (y) = (L-y)\rho_\infty (-g_y) $$ 

L is the length from the "surface" so to speak.

%So that

$$\frac{d P_\infty}{dy} = -\rho_\infty (-g_y) = -\rho_\infty g$$

substitute this back and we get:

$$ u \frac{\partial}{\partial x} v + v \frac{\partial}{\partial y} v  - \nu ( \frac{\partial^2}{\partial x^2} v + \frac{\partial^2}{\partial y^2} \ v ) = - \frac{1}{\rho_0} \frac{\partial P}{\partial y} +g_y$$

$$ u \frac{\partial}{\partial x} v + v \frac{\partial}{\partial y} v  - \nu ( \frac{\partial^2}{\partial x^2} v + \frac{\partial^2}{\partial y^2} \ v ) = - \frac{1}{\rho_0} (-\rho_\infty (-g_y)) +g_y$$

of course $g_y$ is negative, we can simply state

$$g_y = -g$$

$$ u \frac{\partial}{\partial x} v + v \frac{\partial}{\partial y} v  - \nu ( \frac{\partial^2}{\partial x^2} v + \frac{\partial^2}{\partial y^2} \ v ) = - \frac{1}{\rho_0} (-\rho_\infty g) -g$$

Note that the $\rho_0$ here refers to fluid density at the BL, not in the freestream.

The change in $\rho_0$ was small compared to other terms. But in this case, on the RHS, the only driving force for the fluid is this density change, so it cannot be neglected as before.

$$\rho u \frac{\partial}{\partial x} v + \rho v \frac{\partial}{\partial y} v  -\rho \nu ( \frac{\partial^2}{\partial x^2} v + \frac{\partial^2}{\partial y^2} \ v ) = (\rho_\infty g) - \rho g$$

$$\rho u \frac{\partial}{\partial x} v + \rho v \frac{\partial}{\partial y} v  -\rho \nu ( \frac{\partial^2}{\partial x^2} v + \frac{\partial^2}{\partial y^2} \ v ) = (\rho_\infty - \rho)  g$$

%How do u then quantify $\rho_\infty - \rho$?

Here's were the boussinesq approximation comes in, only applies for small density changes due to temperature, not pressure. Constant static pressure is assumed (\cite{bejan2013convection})

$$\rho = \rho_\infty (1- \beta(T-T_\infty))$$

This is a linear relationship for a constant $\beta$ value (thermal expansion coefficient).

$$ \beta  = - \frac{1}{\rho} \left( \frac{\partial \rho}{\partial T} \right)_P$$

When we substitute all this in, we are ready to start deriving the equations.

$$\rho u \frac{\partial}{\partial x} v + \rho v \frac{\partial}{\partial y} v  -\rho \nu ( \frac{\partial^2}{\partial x^2} v + \frac{\partial^2}{\partial y^2} \ v ) = \rho_\infty \beta(T-T_\infty)  g$$

Once again, for the sake of simplification, density changes are not important except that the buoyancy forces are accounted for, therefore,

$$\rho \approx \rho_\infty$$

$$ u \frac{\partial}{\partial x} v +  v \frac{\partial}{\partial y} v  - \nu ( \frac{\partial^2}{\partial x^2} v + \frac{\partial^2}{\partial y^2} \ v ) =  \beta(T-T_\infty)  g$$


This substitution is known as the boussinesq approximation.

First we start with continuity

$$  \frac{\partial}{\partial x} u +  \frac{\partial}{\partial y} v  = 0$$

$$ x = x^* \delta_T$$

$$y=y^* H$$

What's the characteristic velocity scale? We don't have a $u_\infty$ to really look to here...

Okay, so maybe, it's easier to start with energy equations because this is our driving force! In forced convection, the driving force is sort of implicit in the momentum and mass conservation equations (freestream velocity is driving force). 

$$ u \frac{\partial}{\partial x} T + v \frac{\partial }{\partial y} \ T  = \alpha ( \frac{\partial^2}{\partial x^2} T +  \frac{\partial^2 }{\partial y^2} \ T) $$ 

So now, we can nondimensionalise T, x and y. u and v will remain. 

Now combine this with the continuity equation we have two equations with two unknowns (order of magnitude so to speak).

$$ u \frac{1}{\delta_T} \frac{\partial}{\partial x^*} T + v \frac{1}{H} \frac{\partial }{\partial y^*} \ T  = \alpha ( \frac{1}{\delta_T^2} \frac{\partial^2}{\partial (x^*)^2} T + \frac{1}{H^2} \frac{\partial^2 }{\partial (y^*)^2} \ T) $$ 

Now we can also nondimensionalise temperature.

$$\theta= \frac{T-T_\infty}{T_s - T_\infty}$$

$$(T_s - T_\infty) d \theta = d (T - T_\infty) =dT$$

$$ u \frac{1}{\delta_T} \frac{\partial}{\partial x^*} T + v \frac{1}{H} \frac{\partial }{\partial y^*} \ T  = \alpha ( \frac{1}{\delta_T^2} \frac{\partial^2}{\partial (x^*)^2} T + \frac{1}{H^2} \frac{\partial^2 }{\partial (y^*)^2} \ T) $$ 


$$ u \frac{1}{\delta_T} \frac{\partial}{\partial x^*} (T - T_\infty) + v \frac{1}{H} \frac{\partial }{\partial y^*}  (T - T_\infty)  = \alpha ( \frac{1}{\delta_T^2} \frac{\partial^2}{\partial (x^*)^2} (T - T_\infty) + \frac{1}{H^2} \frac{\partial^2 }{\partial (y^*)^2} (T - T_\infty)) $$ 

$$ u \frac{(T_s - T_\infty)}{\delta_T} \frac{\partial}{\partial x^*}\theta + v \frac{(T_s - T_\infty)}{H} \frac{\partial }{\partial y^*}  \theta  = \alpha ( \frac{(T_s - T_\infty)}{\delta_T^2} \frac{\partial^2}{\partial (x^*)^2} \theta + \frac{(T_s - T_\infty)}{H^2} \frac{\partial^2 }{\partial (y^*)^2} \theta) $$ 

Now since $H >> \delta_T$, we can assume that

$$( \frac{(T_s - T_\infty)}{\delta_T^2} \frac{\partial^2}{\partial (x^*)^2} \theta + \frac{(T_s - T_\infty)}{H^2} \frac{\partial^2 }{\partial (y^*)^2} \theta) \approx ( \frac{(T_s - T_\infty)}{\delta_T^2} \frac{\partial^2}{\partial (x^*)^2} \theta )$$

$$ u \frac{(T_s - T_\infty)}{\delta_T} \frac{\partial}{\partial x^*}\theta + v \frac{(T_s - T_\infty)}{H} \frac{\partial }{\partial y^*}  \theta  = \alpha ( \frac{(T_s - T_\infty)}{\delta_T^2} \frac{\partial^2}{\partial (x^*)^2} \theta ) $$ 

And since the nondimensionalised terms are all $\mathcal{O}(1)$

$$\mathcal{O}( u \frac{(T_s - T_\infty)}{\delta_T}) + \mathcal{O}(v \frac{(T_s - T_\infty)}{H}) = \mathcal{O}( \alpha  \frac{(T_s - T_\infty)}{\delta_T^2})$$


So now we can work with the mass balance equations:


$$  \frac{\partial}{\partial x} u +  \frac{\partial}{\partial y} v  = 0$$

$$ x = x^* \delta_T$$

$$y=y^* H$$


$$ \frac{1}{\delta_T} \frac{\partial}{\partial x^* } u + \frac{1}{H} \frac{\partial}{\partial y^*} v  = 0$$

$$ \frac{\mathcal{O}(u)}{\delta_T} \frac{\partial}{\partial x^* } u^* + \frac{\mathcal{O}(v)}{H} \frac{\partial}{\partial y^*} v^*  = 0 $$

$$u=\mathcal{O}(u) u^*$$

$$\mathcal{O}(\frac{u}{\delta_T}) = \mathcal{O}(\frac{v}{H})$$

Where u and v are typical velocity values in the BL. This is thermal BL btw

From this, we find that we cannot just throw out any of the terms here... However, we do know that the x and y velocity terms are on the same order of magnitude. We can then perform elimination to find:

$$\mathcal{O}(  \frac{u}{\delta_T}) + \mathcal{O}( \frac{v}{H}) = \mathcal{O}(   \frac{\alpha}{\delta_T^2})$$

$$2 \mathcal{O}(  \frac{u}{\delta_T}) = \mathcal{O}(   \frac{\alpha}{\delta_T^2})$$

$$2 \mathcal{O}( u ) = \mathcal{O}(\frac{\alpha}{\delta_T})$$

if i eliminate u,

$$2 \mathcal{O}(v ) = \mathcal{O}( \frac{\alpha H}{\delta_T^2})$$

Of course we can easily get rid of the factor of 2, since it is still in the same order of magnitude

$$2 = \mathcal{O}(1)$$

$$ \mathcal{O}( u ) = \mathcal{O}(\frac{\alpha}{\delta_T})$$


$$ \mathcal{O}(v ) = \mathcal{O}( \frac{\alpha H}{\delta_T^2})$$

Now let's move on to the momentum equations

$$ u \frac{\partial}{\partial x} v +  v \frac{\partial}{\partial y} v  - \nu ( \frac{\partial^2}{\partial x^2} v + \frac{\partial^2}{\partial y^2} \ v ) = \beta(T-T_\infty)  g$$



Now here, we only nondimensionalise the position and velocity coordinates, for density and viscosity, while they are changing withing the fluid, we choose not to nondimensionalise as the changes are not as significant as the terms contributing to inertial forces, frictional forces and buoyancy forces.

The easiest is to start with the temperature.

$$ u \frac{\partial}{\partial x} v +  v \frac{\partial}{\partial y} v  -\nu ( \frac{\partial^2}{\partial x^2} v + \frac{\partial^2}{\partial y^2} \ v ) = \beta(T_s-T_\infty) g \theta $$

Then the advection terms,

$$ \mathcal{O}(u) \mathcal{O}(v) \frac{1}{\delta_T} u^* \frac{\partial}{\partial x^*} v^* +  \mathcal{O}(v) \mathcal{O}(v) \frac{1}{H} v^* \frac{\partial}{\partial y^*} v^*  -\nu ( \frac{\partial^2}{\partial x^2} v + \frac{\partial^2}{\partial y^2} \ v ) = \beta(T_s-T_\infty) g \theta $$

Then the frictional (viscous) forces

$$  \mathcal{O}(u) \mathcal{O}(v) \frac{1}{\delta_T} u^* \frac{\partial}{\partial x^*} v^* +  \mathcal{O}(v) \mathcal{O}(v) \frac{1}{H} v^* \frac{\partial}{\partial y^*} v^*  -\nu ( \frac{\mathcal{O}(v)}{\delta_T^2} \frac{\partial^2}{\partial (x^*)^2} v^* + \frac{\mathcal{O}(v)}{H^2} \frac{\partial^2}{\partial (y^*)^2} \ v^* ) = \beta(T_s-T_\infty) g \theta $$

We make an approximation:

$$( \frac{\mathcal{O}(v)}{\delta_T^2} \frac{\partial^2}{\partial (x^*)^2} v^* + \frac{\mathcal{O}(v)}{H^2} \frac{\partial^2}{\partial (y^*)^2} \ v^* ) \approx ( \frac{\mathcal{O}(v)}{\delta_T^2} \frac{\partial^2}{\partial (x^*)^2} v^* )$$

So we are left with:

$$  \mathcal{O}(u) \mathcal{O}(v) \frac{1}{\delta_T} u^* \frac{\partial}{\partial x^*} v^* + \mathcal{O}(v) \mathcal{O}(v) \frac{1}{H} v^* \frac{\partial}{\partial y^*} v^*  -\nu ( \frac{\mathcal{O}(v)}{\delta_T^2} \frac{\partial^2}{\partial (x^*)^2} v^* ) = \beta(T_s-T_\infty) g \theta $$

$$ \frac{\alpha}{\delta_T} \mathcal{O}(v) \frac{1}{\delta_T} u^* \frac{\partial}{\partial x^*} v^* + \mathcal{O}(v) \mathcal{O}(v) \frac{1}{H} v^* \frac{\partial}{\partial y^*} v^*  -\nu ( \frac{\mathcal{O}(v)}{\delta_T^2} \frac{\partial^2}{\partial (x^*)^2} v^* ) = \beta(T_s-T_\infty) g \theta $$


$$ \frac{\alpha}{\delta_T}\frac{\alpha H}{\delta_T^2}\frac{1}{\delta_T} u^* \frac{\partial}{\partial x^*} v^* + \frac{\alpha H}{\delta_T^2} \frac{\alpha H}{\delta_T^2} \frac{1}{H} v^* \frac{\partial}{\partial y^*} v^*  -\nu ( \frac{\frac{\alpha H}{\delta_T^2}}{\delta_T^2} \frac{\partial^2}{\partial (x^*)^2} v^* ) = \beta(T_s-T_\infty) g \theta $$

$$ \frac{\alpha^2 H}{\delta_T^4} u^* \frac{\partial}{\partial x^*} v^* + \frac{\alpha^2 H}{\delta_T^4} v^* \frac{\partial}{\partial y^*} v^*  -\nu ( \frac{\alpha H}{\delta_T^4} \frac{\partial^2}{\partial (x^*)^2} v^* ) = \beta(T_s-T_\infty) g \theta $$

Viscous forces are always significant in laminar BL flow.

Or in its dimensional form,

$$ u \frac{\partial}{\partial x} v +  v \frac{\partial}{\partial y} v  - \nu ( \frac{\partial^2}{\partial x^2} v) = \beta(T-T_\infty)  g$$


Now if we want to nondimensionalise everything, we found the characteristic velocity scaling to be:

$$\mathcal{O}(\frac{u}{\delta_T}) = \mathcal{O}(\frac{v}{H})$$

$$ \mathcal{O}( u ) = \mathcal{O}(\frac{\alpha}{\delta_T})$$
$$ \mathcal{O}(v ) = \mathcal{O}( \frac{\alpha H}{\delta_T^2})$$

Hence, we can use these to nondimensionalise the velocity scales

$$  \mathcal{O}(u) \mathcal{O}(v) \frac{1}{\delta_T} u^* \frac{\partial}{\partial x^*} v^* +  \mathcal{O}(v) \mathcal{O}(v) \frac{1}{H} v^* \frac{\partial}{\partial y^*} v^*  -\nu ( \frac{\mathcal{O}(v)}{\delta_T^2} \frac{\partial^2}{\partial (x^*)^2} v^* ) = \beta(T_s-T_\infty) g \theta $$

$$  \frac{\alpha}{\delta_T} \frac{\alpha H}{\delta_T^2} \frac{1}{\delta_T} u^* \frac{\partial}{\partial x^*} v^* +  \frac{\alpha H}{\delta_T^2} \frac{\alpha H}{\delta_T^2} \frac{1}{H} v^* \frac{\partial}{\partial y^*} v^*  -\nu ( \frac{\alpha H}{\delta_T^2} \frac{1}{\delta_T^2} \frac{\partial^2}{\partial (x^*)^2} v^* ) = \beta(T_s-T_\infty) g \theta $$

Now collecting terms together,

$$  \frac{\alpha^2 H}{\delta_T^4}  u^* \frac{\partial}{\partial x^*} v^* +  \frac{\alpha^2 H}{\delta_T^4} v^* \frac{\partial}{\partial y^*} v^*  -\nu ( \frac{\alpha H}{\delta_T^4} \frac{\partial^2}{\partial (x^*)^2} v^* ) = \beta(T_s-T_\infty) g \theta $$


we divide both sides by, $\frac{\alpha^2 H}{\delta_T^4}$

$$   u^* \frac{\partial}{\partial x^*} v^* +    v^* \frac{\partial}{\partial y^*} v^*  -  ( \frac{\nu}{\alpha} \frac{\partial^2}{\partial (x^*)^2} v^* ) = \frac{\delta_T^4}{\alpha^2 H} \beta(T_s-T_\infty) g \theta $$

We introduce the prandtl number,

$$   u^* \frac{\partial}{\partial x^*} v^* +    v^* \frac{\partial}{\partial y^*} v^*  -  ( Pr \frac{\partial^2}{\partial (x^*)^2} v^* ) = \frac{\delta_T^4}{\alpha^2 H} \beta(T_s-T_\infty) g \theta $$

Also we notice the ratio $\frac{\delta_T}{H}$ should be important, it describes the relationship between thermal BL thickness and wall height.

$$   u^* \frac{\partial}{\partial x^*} v^* +    v^* \frac{\partial}{\partial y^*} v^*  -  ( Pr \frac{\partial^2}{\partial (x^*)^2} v^* ) = \frac{\delta_T^4}{\alpha^2 H^4} \beta(T_s-T_\infty) H^3 g \theta $$



$$   u^* \frac{\partial}{\partial x^*} v^* +    v^* \frac{\partial}{\partial y^*} v^*  -  ( Pr \frac{\partial^2}{\partial (x^*)^2} v^* ) = \frac{\delta_T^4}{ H^4} \frac{\beta(T_s-T_\infty) H^3 g}{\alpha^2}  \theta $$

Divide throughout by Pr,

$$ \frac{1}{Pr} \left[ u^* \frac{\partial}{\partial x^*} v^* +    v^* \frac{\partial}{\partial y^*} v^* \right] -   \frac{\partial^2}{\partial (x^*)^2} v^*  = \frac{\delta_T^4}{ H^4} \frac{\beta(T_s-T_\infty) H^3 g}{\alpha^2} \frac{1}{Pr} \theta $$

$$ \frac{1}{Pr} \left[ u^* \frac{\partial}{\partial x^*} v^* +    v^* \frac{\partial}{\partial y^*} v^* \right] -   \frac{\partial^2}{\partial (x^*)^2} v^*  = \frac{\delta_T^4}{ H^4} \frac{\beta(T_s-T_\infty) H^3 g}{\alpha^2} \frac{\alpha	}{\nu} \theta $$

$$ \frac{1}{Pr} \left[ u^* \frac{\partial}{\partial x^*} v^* +    v^* \frac{\partial}{\partial y^*} v^* \right] -   \frac{\partial^2}{\partial (x^*)^2} v^*  = \frac{\delta_T^4}{ H^4} \frac{\beta(T_s-T_\infty) H^3 g}{\alpha \nu}  \theta $$


Here we see the occurence of an important number:

$$Ra_H =  \frac{\beta(T_s-T_\infty) H^3 g}{\alpha \nu}$$



This is called the Rayleigh number. Which is THE important term for natural convection, it is dependent on both thermal diffusivity and momentum diffusivity. 

$$ \frac{1}{Pr} \left[ u^* \frac{\partial}{\partial x^*} v^* +    v^* \frac{\partial}{\partial y^*} v^* \right] -   \frac{\partial^2}{\partial (x^*)^2} v^*  = \frac{\delta_T^4}{ H^4} Ra_H  \theta $$


It mismashes all the effects in one term. Both thermal and momentum diffusivity, so it's not quite like the Reynold's number.

We can see that Pr is important in whether we consider advection of momentum an important effect. In fact $\frac{1}{Pr}$ is important in this equation. We can remove these terms, but what are they small in comparision to? We need another term with $\frac{1}{Pr}$ coefficient to compare it to.

Normally if Prandtl number is really big, and the Rayleigh number is big, this comparison is not needed. But if the prandtl number is small, we got a problem. So we'll need to do comparisons another way.


$$ \frac{1}{Pr} \left[ u^* \frac{\partial}{\partial x^*} v^* +    v^* \frac{\partial}{\partial y^*} v^* \right] -   \frac{\partial^2}{\partial (x^*)^2} v^*  = \frac{\delta_T^4}{ H^4} \frac{\beta(T_s-T_\infty) H^3 g}{\alpha \nu}  \theta $$


note: Pr is very important,
It describes the relationship between momentum BL thickness and thermal BL thickness
$$\frac{\delta_p}{\delta_t} = f(Pr)$$

If Pr low, (liquid metals)
let's try this trick

$$ \frac{1}{Pr} \left[ u^* \frac{\partial}{\partial x^*} v^* +    v^* \frac{\partial}{\partial y^*} v^* \right] -   \frac{\partial^2}{\partial (x^*)^2} v^*  = \frac{\delta_T^4}{ H^4} \frac{\beta(T_s-T_\infty) H^3 g}{\alpha \nu}  \frac{Pr}{Pr} \theta $$


$$ \frac{1}{Pr} \left[ u^* \frac{\partial}{\partial x^*} v^* +    v^* \frac{\partial}{\partial y^*} v^* \right] -   \frac{\partial^2}{\partial (x^*)^2} v^*  = \frac{\delta_T^4}{ H^4} \frac{\beta(T_s-T_\infty) H^3 g}{\alpha^2}  \frac{1}{Pr} \theta $$

Only within thermal BL, advection for liquid metals or low Pr fluids, we can neglect momentum transport away from thermal BL through viscous action

$$ \frac{1}{Pr} \left[ u^* \frac{\partial}{\partial x^*} v^* +    v^* \frac{\partial}{\partial y^*} v^* \right]  = \frac{\delta_T^4}{ H^4} \frac{\beta(T_s-T_\infty) H^3 g}{\alpha^2}  \frac{1}{Pr} \theta $$

$$ \left[ u^* \frac{\partial}{\partial x^*} v^* +    v^* \frac{\partial}{\partial y^*} v^* \right]  = \frac{\delta_T^4}{ H^4} \frac{\beta(T_s-T_\infty) H^3 g}{\alpha^2}   \theta $$


Now we find this new dimensionless number(\cite{bejan2013convection}):

$$Bo=\frac{\beta(T_s-T_\infty) H^3 g}{\alpha^2}=Ra_H Pr$$

This is called the boussinesq number. Important for fluids with low Pr, so that the viscous term can then be neglected.


$$ u \frac{(T_s - T_\infty)}{\delta_T} \frac{\partial}{\partial x^*}\theta + v \frac{(T_s - T_\infty)}{H} \frac{\partial }{\partial y^*}  \theta  = \alpha ( \frac{(T_s - T_\infty)}{\delta_T^2} \frac{\partial^2}{\partial (x^*)^2} \theta ) $$ 

\section{solution procedure - integral}

Now to solve the equation. First we always need to know what we are solving for, otherwise we will go in circles.

We have some questions of interest.

\begin{itemize}
\item What is the Nu (heat transfer coeff)?
\item Velocity profile
\item BL thickness of thermal and momentum BL
\end{itemize}

Remember, we may not want to solve the equation in totality, but rather go for special cases, eg. high or low Pr to make life easier for ourselves.

For example, in high Pr fluids, we want to just neglect the momentum advection terms, making life easier for ourselves. 

\subsection{High Pr fluids}

In this case, very simply, we eliminate the momentum advection terms.

$$ \frac{1}{Pr} \left[ u^* \frac{\partial}{\partial x^*} v^* +    v^* \frac{\partial}{\partial y^*} v^* \right] -   \frac{\partial^2}{\partial (x^*)^2} v^*  = \frac{\delta_T^4}{ H^4} Ra_H \theta $$

$$  -   \frac{\partial^2}{\partial (x^*)^2} v^*  = \frac{\delta_T^4}{ H^4} Ra_H \theta $$

From this we can see that:

$$ \frac{\delta_T^4}{ H^4} Ra_H =\mathcal{O}(1)$$
(remember all dimensionless terms are of the order of magnitude of 1)

$$\delta_T = \mathcal{O} (H Ra_H^{-\frac{1}{4}})$$

Another way to write this is:
$$\delta_T = \mathcal{O}(1) H Ra_H^{-\frac{1}{4}}$$
$$\delta_T = \mathcal{O}(1) H (\frac{\beta(T_s-T_\infty) H^3 g}{\alpha \nu})^{-\frac{1}{4}}$$

$$\delta_T = \mathcal{O}(1) H^{1/4} (\frac{\beta(T_s-T_\infty) g}{\alpha \nu})^{-\frac{1}{4}}$$


Based on this, we can guess the velocity profile. Given that

$$v = \mathcal{O}(\frac{aH}{\delta_T^2})$$
$$v = \mathcal{O}(1)\frac{aH}{\delta_T^2}$$

$$...$$

$$v \sim H^{0.5}$$

What about the Nusselt number or heat transfer coeffcient?

We need to then look at our energy equations:

$$ u \frac{(T_s - T_\infty)}{\delta_T} \frac{\partial}{\partial x^*}\theta + v \frac{(T_s - T_\infty)}{H} \frac{\partial }{\partial y^*}  \theta  = \alpha ( \frac{(T_s - T_\infty)}{\delta_T^2} \frac{\partial^2}{\partial (x^*)^2} \theta ) $$ 


The heat transfer in the x direction is governed by,

$$\frac{\partial T}{\partial x} = \mathcal{O}( u \frac{(T_s - T_\infty)}{\delta_T})$$

Recall:

$$u  = \mathcal{O}(\frac{\alpha}{\delta_T})$$

$$\frac{\partial T}{\partial x} = \mathcal{O}( \frac{\alpha}{\delta_T} \frac{(T_s - T_\infty)}{\delta_T})$$

$$\frac{\partial T}{\partial x} = \mathcal{O}( \alpha \frac{(T_s - T_\infty)}{\delta_T^2})$$

$$v = \mathcal{O}(\frac{aH}{\delta_T^2})$$

$$\frac{\partial T}{\partial y} = \mathcal{O}(1) v \frac{(T_s - T_\infty)}{H}  $$

$$\frac{\partial T}{\partial y} = \mathcal{O}(1) \frac{aH}{\delta_T^2} \frac{(T_s - T_\infty)}{H}  $$

$$\frac{\partial T}{\partial y} = \mathcal{O}(1) \frac{a(T_s - T_\infty)}{\delta_T^2}  $$


$$ u^* \frac{\partial}{\partial x^*}\theta + v^* \frac{\partial }{\partial y^*}  \theta  = \frac{\partial^2}{\partial (x^*)^2} \theta  $$ 


With this we can see how the temperature gradient scales as we go up the wall. 

Why is all this important? Because we assume a velocity or temperature profile shape when we do integral solution analysis. 

So how can we get a temperature profile across x? It's more or less qualitative. Just like for the original von karman solution, a polynomial solution was assumed.

We can try looking at the energy equations and come up with some qualitative shapes:

Consider the energy equation, hold y constant. 

$$ u^* \frac{\partial}{\partial x^*}\theta + v^* \frac{\partial }{\partial y^*}  \theta  = \frac{\partial^2}{\partial (x^*)^2} \theta  $$ 


As we go further away from the wall, intuitively u and v increase. Or more correctly advection increases. Up to some certain maximum point, and then falls off.

For temperature, there is conduction effect but with some advection which effectively acts like a heat sink. 

In Bejan's textbook, the recommended temp profile is(\cite{bejan2013convection}):

$$T-T_\infty = (T_s -T_\infty) \exp (-\frac{x}{\delta_T})$$

$$\theta = \exp (-\frac{x}{\delta_T})$$

If we were to substitute this back,

$$ \frac{1}{Pr} \left[ u^* \frac{\partial}{\partial x^*} v^* +    v^* \frac{\partial}{\partial y^*} v^* \right] -   \frac{\partial^2}{\partial (x^*)^2} v^*  = \frac{\delta_T^4}{ H^4} Ra_H \theta $$

$$-   \frac{\partial^2}{\partial (x^*)^2} v^*  = \frac{\delta_T^4}{ H^4} Ra_H \theta $$

$$-   \frac{\partial^2}{\partial (x^*)^2} v^*  = \frac{\delta_T^4}{ H^4} Ra_H \exp (-\frac{x}{\delta_T}) $$

$$-   \frac{\partial^2}{\partial (x^*)^2} v^*  = \mathcal{O}(1) \exp (-\frac{x}{\delta_T}) $$

$$v^* = f(\exp (-\frac{x^*}{\delta_T}))$$

In Bejan's textbook, the recommended velocity profile is (\cite{bejan2013convection}):


$$v (x,y) = v_Y (y) \exp (-\frac{x}{\delta_p}) (1- \exp (-\frac{x}{\delta_T}))$$

Now this is just an estimate and assumption, based on the rough shapes of the BL, there are no exact reasoning why there must be exponential decay shapes. 

BUT.... one can think of it this way, consider the control volume in the BL,

For exponential equations, we will have the form
$$\frac{dT}{dx} = -kT$$


There's conduction in and out (right term), and advection in and out (left terms)

along x axis, velocity is likely to go in negative x direction. Fluid flows up, has to come in from somewhere. Also will draw in cold fluid and release hot fluid at temperature T.

Along y axis, velocity goes in positive y direction, again, draw in cold fluid, release hot fluid at temperature T.

Basically, we do have a heat loss term in the form kT. 

Note that there is a heat flux term 

Now for integral solutions, we need to integrate momentum and energy equations across BL in x direction:

$$ u \frac{\partial}{\partial x} v +  v \frac{\partial}{\partial y} v  -\nu ( \frac{\partial^2}{\partial x^2} v ) = \beta(T_s-T_\infty) g \theta $$

$$ \int_0^{x_\infty}  dx\ u \frac{\partial}{\partial x} v  + \int_0^{x_\infty}  dx \  v \frac{\partial}{\partial y} v  - \int_0^{x_\infty}  dx \ \nu ( \frac{\partial^2}{\partial x^2} v ) =  \int_0^{x_\infty}  dx \ \beta(T_s-T_\infty) g \theta $$

The choice of $x_\infty$ is such that the conditions are close enough to freestream. Ie $u = u_\infty = 0 $, $T=T_\infty$.

We can do the same for energy equation

$$ u \frac{\partial}{\partial x} T + v  \frac{\partial }{\partial y}  T  = \alpha (  \frac{\partial^2}{\partial x^2} T ) $$ 

$$  \int_0^{x_\infty}  dx\ u \frac{\partial}{\partial x} T +  \int_0^{x_\infty}  dx\  v  \frac{\partial }{\partial y}  T  =  \int_0^{x_\infty}  dx\  \alpha (  \frac{\partial^2}{\partial x^2} T ) $$ 

We can start by sorting out the non advection terms first, these are easier. We assume the thermal diffusivity, expansion and momentum diffusivity take on average values at film temperature $T_f = \frac{T_s+T_\infty}{2}$



For the energy equation, we can combine back the terms:

$$ \frac{\partial}{\partial x}(uT)   + \frac{\partial}{\partial y}(vT) -T( \frac{\partial}{\partial y}v + \frac{\partial}{\partial x}u) = \alpha (\frac{\partial^2 }{\partial x^2} \ T ) $$ 

Fortunately,

$$\frac{\partial u}{\partial x} + \frac{\partial v}{\partial y}= 0$$

$$ \frac{\partial}{\partial x}(uT)   + \frac{\partial}{\partial y}(vT)  = \alpha (\frac{\partial^2 }{\partial x^2} \ T ) $$ 




$$\int_0^{x_\infty}  dx\   \frac{\partial}{\partial x}(uT)   + \int_0^{x_\infty}  dx\  \frac{\partial}{\partial y}(vT)  = \int_0^{x_\infty}  dx\  \alpha (\frac{\partial^2 }{\partial x^2} \ T ) $$ 


$$ (uT)|_\infty - (uT)|_0   + \int_0^{x_\infty}  dx\  \frac{\partial}{\partial y}(vT)  = \int_0^{x_\infty}  dx\  \alpha (\frac{\partial^2 }{\partial x^2} \ T ) $$ 

$$  \int_0^{x_\infty}  dx\  \frac{\partial}{\partial y}(vT)  = \int_0^{x_\infty}  dx\  \alpha (\frac{\partial^2 }{\partial x^2} \ T ) $$ 


$$  \int_0^{x_\infty}  dx\  \frac{\partial}{\partial y}(vT)  = \alpha ( \frac{\partial T }{\partial x}|_\infty - \frac{\partial T }{\partial x}|_0 )   $$ 

$$  \int_0^{x_\infty}  dx\  \frac{\partial}{\partial y}(vT)  = \alpha ( - \frac{\partial T }{\partial x}|_0 )   $$ 

And we'll have to use Leibniz's rule again...


$$ \frac{d}{dx} \left( \int_{y1=a(x)}^{y2=b(x)} f(x,y) dy \right)= f(x,y=b(x)) \frac{d}{dx}b(x) - f(x,y=a(x)) \frac{d}{dx} a(x)  +  \int_{y1=a(x)}^{y2=b(x)} \frac{\partial}{\partial x} f(x,y) dy $$

Note that our x and y is swapped around

$$ \frac{d}{dy} \left( \int_{x1=a(y)}^{x2=b(y)} f(y,x) dx \right)= f(y,x=b(y)) \frac{d}{dy}b(y) - f(y,x=a(y)) \frac{d}{dy} a(y)  +  \int_{x1=a(y)}^{x2=b(y)} \frac{\partial}{\partial y} f(y,x) dx $$

$$x1=0 \ ; a(y) = 0$$
$$x2= \delta_p (y) \ ; b(y) = \delta_p$$

$$f(y,x) = vT$$
Substitute into leibniz's rule:

$$ \frac{d}{dy} \left( \int_{x1=0}^{x2=\delta_p(y)} vT dx \right)= vT (y,x=\delta_p(y)) \frac{d}{dy}\delta_p(y) - vT (y,x=0) \frac{d}{dy} 0  +  \int_{x1=0}^{x2=\delta_p(y)} \frac{\partial}{\partial y} vT dx $$


$$ \frac{d}{dy} \left( \int_{x1=0}^{x2=\delta_p(y)} vT dx \right)= vT (y,x=\delta_p(y)) \frac{d}{dy}\delta_p(y)  +  \int_{x1=0}^{x2=\delta_p(y)} \frac{\partial}{\partial y} vT dx $$

$$v=0 \ at \ \delta_p$$

$$ \frac{d}{dy} \left( \int_{x1=0}^{x2=\delta_p(y)} vT dx \right)=   \int_{x1=0}^{x2=\delta_p(y)} \frac{\partial}{\partial y} vT dx $$

Change some variables to make it easier to nondimensionalise 

$$ \frac{d}{dy} \left( \int_{x1=0}^{x2=\delta_p(y)} v(T-T_\infty) dx \right)=   \int_{x1=0}^{x2=\delta_p(y)} \frac{\partial}{\partial y} v(T-T_\infty) dx = \int_{x1=0}^{x2=\delta_p(y)} \frac{\partial}{\partial y} vT dx$$

subs back in energy equation:

$$  \int_0^{x_\infty}  dx\  \frac{\partial}{\partial y}(vT)  = \alpha ( - \frac{\partial T }{\partial x}|_0 )   $$ 

$$   \frac{d}{dy} \left( \int_{x1=0}^{x2=\delta_p(y)} v(T-T_\infty) dx \right) = \alpha ( - \frac{\partial T }{\partial x}|_0 )   $$ 

$$   \frac{d}{dy} \left( \int_{0}^{\delta_p(y)} v(T_\infty - T) dx \right) = \alpha \frac{\partial T }{\partial x}|_{x=0}    $$ 

Now we are done with energy equation, let's move on to momentum...

$$ \int_0^{x_\infty}  dx\ u \frac{\partial}{\partial x} v  + \int_0^{x_\infty}  dx \  v \frac{\partial}{\partial y} v  - \int_0^{x_\infty}  dx \ \nu ( \frac{\partial^2}{\partial x^2} v ) =  \int_0^{x_\infty}  dx \ \beta(T_s-T_\infty) g \theta $$

$$ \int_0^{x_\infty}  dx\ (\frac{\partial }{\partial x} (uv) - v \frac{\partial u}{\partial x})  + \int_0^{x_\infty}  dx \  v \frac{\partial}{\partial y} v  - \int_0^{x_\infty}  dx \ \nu ( \frac{\partial^2}{\partial x^2} v ) =  \int_0^{x_\infty}  dx \ \beta(T_s-T_\infty) g \theta $$


For advection terms, we want to eliminate u by considering 2D continuity equation.

$$\frac{\partial u}{\partial x} + \frac{\partial v}{\partial y}= 0$$

$$ \frac{\partial v}{\partial y}= -\frac{\partial u}{\partial x}$$

Substitute back in To get:


$$ \int_0^{x_\infty}  dx\ (\frac{\partial }{\partial x} (uv) + v \frac{\partial v}{\partial y})  + \int_0^{x_\infty}  dx \  v \frac{\partial}{\partial y} v  - \int_0^{x_\infty}  dx \ \nu ( \frac{\partial^2}{\partial x^2} v ) =  \int_0^{x_\infty}  dx \ \beta(T_s-T_\infty) g \theta $$



$$ \int_0^{x_\infty}  dx\ (\frac{\partial }{\partial x} (uv) )  + \int_0^{x_\infty}  dx \ 2 v \frac{\partial}{\partial y} v  - \int_0^{x_\infty}  dx \ \nu ( \frac{\partial^2}{\partial x^2} v ) =  \int_0^{x_\infty}  dx \ \beta(T_s-T_\infty) g \theta $$

$$ \int_0^{x_\infty}  dx\ (\frac{\partial }{\partial x} (uv) )  + \int_0^{x_\infty}  dx \  \frac{\partial}{\partial y} v^2  - \int_0^{x_\infty}  dx \ \nu ( \frac{\partial^2}{\partial x^2} v ) =  \int_0^{x_\infty}  dx \ \beta(T_s-T_\infty) g \theta $$

The first integral disappears because of no slip and freestream conditions

$$ \int_0^{x_\infty}  dx\ (\frac{\partial }{\partial x} (uv) ) = - uv|_{wall} + uv|_{freestream} = 0 - 0 = 0$$

So we are left with:

$$ \int_0^{x_\infty}  dx \  \frac{\partial}{\partial y} v^2  - \int_0^{x_\infty}  dx \ \nu ( \frac{\partial^2}{\partial x^2} v ) =  \int_0^{x_\infty}  dx \ \beta(T_s-T_\infty) g \theta $$

We'll need to have a go with leibniz's rule

$$\frac{d}{dx} \left( \int_{y1=a(x)}^{y2=b(x)} f(x,y) dy \right)= f(x,y=b(x)) \frac{d}{dx}b(x) - f(x,y=a(x)) \frac{d}{dx} a(x)  +  \int_{y1=a(x)}^{y2=b(x)} \frac{\partial}{\partial x} f(x,y) dy $$

Note that our x and y is swapped around

$$ \frac{d}{dy} \left( \int_{x1=a(y)}^{x2=b(y)} f(y,x) dx \right)= f(y,x=b(y)) \frac{d}{dy}b(y) - f(y,x=a(y)) \frac{d}{dy} a(y)  +  \int_{x1=a(y)}^{x2=b(y)} \frac{\partial}{\partial y} f(y,x) dx $$

$$x1=0 \ ; a(y) = 0$$
$$x2= \delta_p (y) \ ; b(y) = \delta_p$$

$$f(y,x) = v^2$$
Substitute into leibniz's rule:

$$ \frac{d}{dy} \left( \int_{x1=0}^{x2=\delta_p(y)} v^2 dx \right)= v^2 (y,x=\delta_p(y)) \frac{d}{dy}\delta_p(y) - v^2 (y,x=0) \frac{d}{dy} 0  +  \int_{x1=0}^{x2=\delta_p(y)} \frac{\partial}{\partial y} v^2 dx $$

$$ \frac{d}{dy} \left( \int_{x1=0}^{x2=\delta_p(y)} v^2 dx \right)=   \int_{x1=0}^{x2=\delta_p(y)} \frac{\partial}{\partial y} v^2 dx $$


$$ \frac{\partial}{\partial y} \int_0^{x_\infty}  dx \   v^2  - \int_0^{x_\infty}  dx \ \nu ( \frac{\partial^2}{\partial x^2} v ) =  \int_0^{x_\infty}  dx \ \beta(T_s-T_\infty) g \theta $$

$$ \frac{\partial}{\partial y} \int_0^{\delta_p}  dx \   v^2  - \int_0^{\delta_p}  dx \ \nu ( \frac{\partial^2}{\partial x^2} v ) =  \int_0^{\delta_p}  dx \ \beta(T_s-T_\infty) g \theta $$


The second term is easy:

$$ \int_0^{\delta_p}  dx \ \nu ( \frac{\partial^2}{\partial x^2} v ) = \nu (\frac{\partial v}{\partial x}|_{\delta_p} - \frac{\partial v}{\partial x}|_{wall} )=\nu ( - \frac{\partial v}{\partial x}|_{wall} )$$

Subs back in:

$$ \frac{\partial}{\partial y} \int_0^{\delta_p}  dx \   v^2  - \nu ( - \frac{\partial v}{\partial x}|_{wall} ) =  \int_0^{\delta_p}  dx \ \beta(T_s-T_\infty) g \theta $$


For the last term:
$$\theta (T_s - T_\infty) = (T - T_\infty)$$

$$ \int_0^{\delta_p}  dx \ \beta(T_s-T_\infty) g \theta =g \beta \int_0^{\delta_p}  dx  (T - T_\infty)$$

Substitute back in:

$$ \frac{\partial}{\partial y} \int_0^{\delta_p}  dx \   v^2  - \nu ( - \frac{\partial v}{\partial x}|_{wall} ) =  g \beta \int_0^{\delta_p}  dx  (T - T_\infty) $$

\subsubsection{BL equation integral soln}

The final form of the BL equation is (\cite{bejan2013convection}):

$$\frac{d}{dy} \int_0^X v^2 dx = - \nu (\frac{\partial v}{\partial x})|_{x=0} + g\beta \int_0^X (T-T_\infty) dx$$

$$\frac{d}{dy} \int_0^X v (T_\infty - T) dx = \alpha (\frac{\partial T}{\partial x})|_{x=0}$$

X is an x value big enough where all values T, u and v are in their freestream values.

\subsubsection{Substitution of velocity profiles and temperature profiles into integral soln}

Now we can start to try and solve this, substitute

$$T-T_\infty = (T_s -T_\infty) \exp (-\frac{x}{\delta_T})$$
$$v (x,y) = v_Y (y) \exp (-\frac{x}{\delta_p}) (1- \exp (-\frac{x}{\delta_T}))$$


For momentum equation first,

$$\frac{d}{dy} \int_0^X v^2 dx = - \nu (\frac{\partial v}{\partial x})|_{x=0} + g\beta \int_0^X (T-T_\infty) dx$$

$$v^2 (x,y) = v_Y^2 (y) [\exp (-\frac{x}{\delta_p})]^2 [(1- \exp (-\frac{x}{\delta_T}))]^2$$

$$v^2 (x,y) = v_Y^2 (y) [\exp (-\frac{2x}{\delta_p})] [1 - 2 \exp (-\frac{x}{\delta_T}) + \exp (-\frac{2x}{\delta_T})]$$

$$v^2 (x,y) = v_Y^2 (y)  [\exp (-\frac{2x}{\delta_p}) - 2 \exp (-\frac{2x}{\delta_p}) \exp (-\frac{x}{\delta_T}) + \exp (-\frac{2x}{\delta_p}) \exp (-\frac{2x}{\delta_T})]$$

$$v^2 (x,y) = v_Y^2 (y)  [\exp (-\frac{2x}{\delta_p}) - 2 \exp (-x(\frac{2}{\delta_p} + \frac{1}{\delta_T}))  + \exp (-x(\frac{2}{\delta_p}+\frac{2}{\delta_T})) ]$$

$$\int_0^X v^2 dx = v_Y^2 (y) [ \int_0^X dx \  \exp (-\frac{2x}{\delta_p}) - \int_0^X dx \  2 \exp (-x(\frac{2}{\delta_p} + \frac{1}{\delta_T}))  + \int_0^X dx \  \exp (-x(\frac{2}{\delta_p}+\frac{2}{\delta_T}) ] $$

Before we continue, combine into single fraction...

$$(\frac{2}{\delta_p} + \frac{1}{\delta_T}) = \frac{2\delta_T + \delta_p}{\delta_T \delta_p}$$

$$(\frac{2}{\delta_p} + \frac{2}{\delta_T}) = \frac{2\delta_T +2 \delta_p}{\delta_T \delta_p}$$

now we are ready:


$$\int_0^X v^2 dx = v_Y^2 (y) [  \frac{-\delta_p}{2} \exp (-\frac{2x}{\delta_p}) - \frac{-2\delta_p \delta_T}{2\delta_T + \delta_p} \exp (-x(\frac{2}{\delta_p} + \frac{1}{\delta_T}))  + \frac{-\delta_P \delta_T}{2 \delta_T + 2\delta_p}  \exp (-x(\frac{2}{\delta_p}+\frac{2}{\delta_T}) ]|_{x=0}^{x=X} $$

$$\int_0^X v^2 dx = v_Y^2 (y) [  \frac{-\delta_p}{2} \exp (-\frac{2x}{\delta_p}) + \frac{2\delta_p \delta_T}{2\delta_T + \delta_p} \exp (-x(\frac{2}{\delta_p} + \frac{1}{\delta_T}))  - \frac{\delta_P \delta_T}{2 \delta_T + 2\delta_p}  \exp (-x(\frac{2}{\delta_p}+\frac{2}{\delta_T}) ]|_{x=0}^{x=X} $$


First evaluate at $x=X$ where X is such that velocity will be at freestream value.

When x is large, the terms in the bracket drop to near 0.

$$[  \frac{-\delta_p}{2} \exp (-\frac{2X}{\delta_p}) + \frac{2\delta_p \delta_T}{2\delta_T + \delta_p} \exp (-X(\frac{2}{\delta_p} + \frac{1}{\delta_T}))  - \frac{\delta_P \delta_T}{2 \delta_T + 2\delta_p}  \exp (-X(\frac{2}{\delta_p}+\frac{2}{\delta_T}) ] \approx 0$$

But we cannot just get rid of it yet... you have to compare it to something else.

let's also evaluate it at x=0,

$$[  \frac{-\delta_p}{2}  + \frac{2\delta_p \delta_T}{2\delta_T + \delta_p} - \frac{\delta_P \delta_T}{2 \delta_T + 2\delta_p}  ]$$

The integral evaluates to:

$$  \frac{-\delta_p}{2} \exp (-\frac{2X}{\delta_p}) + \frac{2\delta_p \delta_T}{2\delta_T + \delta_p} \exp (-X(\frac{2}{\delta_p} + \frac{1}{\delta_T}))  - \frac{\delta_P \delta_T}{2 \delta_T + 2\delta_p}  \exp (-X(\frac{2}{\delta_p}+\frac{2}{\delta_T}) ] $$

$$ - [  \frac{-\delta_p}{2}  + \frac{2\delta_p \delta_T}{2\delta_T + \delta_p} - \frac{\delta_P \delta_T}{2 \delta_T + 2\delta_p}  ]$$

$$\approx - [  \frac{-\delta_p}{2}  + \frac{2\delta_p \delta_T}{2\delta_T + \delta_p} - \frac{\delta_P \delta_T}{2 \delta_T + 2\delta_p}  ]$$

$$=  [  \frac{\delta_p}{2}  - \frac{2\delta_p \delta_T}{2\delta_T + \delta_p} + \frac{\delta_P \delta_T}{2 \delta_T + 2\delta_p}  ]$$

$$=  [  \frac{\delta_p(2\delta_T + \delta_p)(2 \delta_T + 2\delta_p)}{2(2\delta_T + \delta_p)(2 \delta_T + 2\delta_p)}  - \frac{2\delta_p \delta_T(2)(2 \delta_T + 2\delta_p)}{2(2\delta_T + \delta_p)(2 \delta_T + 2\delta_p)} + \frac{2(2\delta_T+\delta_p) \delta_P \delta_T}{2(2\delta_T+\delta_p)(2 \delta_T + 2\delta_p)}  ]$$

$$= \frac{\delta_p(2\delta_T + \delta_p)(2 \delta_T + 2\delta_p)-4\delta_p \delta_T(2 \delta_T + 2\delta_p)+2(2\delta_T+\delta_p) \delta_P \delta_T}{2(2\delta_T+\delta_p)(2 \delta_T + 2\delta_p)}$$

$$= \frac{\delta_p(2\delta_T + \delta_p)( \delta_T + \delta_p)-4\delta_p \delta_T( \delta_T + \delta_p)+(2\delta_T+\delta_p) \delta_P \delta_T}{2(2\delta_T+\delta_p)( \delta_T + \delta_p)}$$

To make life easier, we introduce this ratio (\cite{bejan2013convection})

$$q = \frac{\delta_p	}{\delta_T}$$

divide top and bottom by $\delta_T^2$

$$= \frac{\delta_p(2+ q)( \delta_T + \delta_p)-4\delta_p( \delta_T + \delta_p)+(2\delta_T+\delta_p) \delta_P }{2(2+q)( \delta_T + \delta_p)}$$

$$= \frac{\delta_p(2+ q)(1+q)-4\delta_p(1+q)+(2+q) \delta_P }{2(2+q)(1+ q)}$$

$$=\delta_p \frac{(2+ q)(1+q)-4(1+q)+(2+q)  }{2(2+q)(1+ q)}$$

$$=\delta_p \frac{(2+ q)(1+q)-4(1+q)+(2+q)  }{2(2+q)(1+ q)}$$
$$=\delta_p \frac{q^2  }{2(2+q)(1+ q)}$$

substitute into integral:

$$\int_0^X v^2 dx = v_Y^2 (y) [  \frac{-\delta_p}{2} \exp (-\frac{2x}{\delta_p}) + \frac{2\delta_p \delta_T}{2\delta_T + \delta_p} \exp (-x(\frac{2}{\delta_p} + \frac{1}{\delta_T}))  - \frac{\delta_P \delta_T}{2 \delta_T + 2\delta_p}  \exp (-x(\frac{2}{\delta_p}+\frac{2}{\delta_T}) ]|_{x=0}^{x=X} $$


$$\int_0^X v^2 dx = v_Y^2 (y) \delta_p \frac{q^2  }{2(2+q)(1+ q)} $$

subs back into equation:

$$\frac{d}{dy} \left[  \frac{v_Y^2 (y) \delta_p q^2  }{2(2+q)(1+ q)} \right] = - \nu (\frac{\partial v}{\partial x})|_{x=0} + g\beta \int_0^X (T-T_\infty) dx$$

Let's do the viscosity term 

$$\frac{\partial v}{\partial x} = \frac{\partial}{\partial x}( v_Y (y) \exp (-\frac{x}{\delta_p}) (1- \exp (-\frac{x}{\delta_T}))) $$

$$ =v_Y (y) \frac{\partial}{\partial x}(  \exp (-\frac{x}{\delta_p}) (1- \exp (-\frac{x}{\delta_T}))) $$

$$ =v_Y (y) \frac{\partial}{\partial x}(\exp (-\frac{x}{\delta_p})- \exp (-\frac{x}{\delta_p}) \exp (-\frac{x}{\delta_T}))) $$

$$ =v_Y (y) \frac{\partial}{\partial x}(\exp (-\frac{x}{\delta_p})- \exp (-x (\frac{1}{\delta_p}+\frac{1}{\delta_T})) ) $$

$$ =v_Y (y) (\frac{-1}{\delta_p} \exp (-\frac{x}{\delta_p})- \frac{-1(\delta_T+\delta_p)}{\delta_T \delta_p} \exp (-x (\frac{1}{\delta_p}+\frac{1}{\delta_T})) ) $$

$$ =v_Y (y) (-\frac{1}{\delta_p} \exp (-\frac{x}{\delta_p})+ \frac{(\delta_T+\delta_p)}{\delta_T \delta_p} \exp (-x (\frac{1}{\delta_p}+\frac{1}{\delta_T})) ) $$

Now evaluate at x=0,

$$(\frac{\partial v}{\partial x})|_{x=0} =v_Y (y) (-\frac{1}{\delta_p} + \frac{(\delta_T+\delta_p)}{\delta_T \delta_p}  ) $$

$$=(\frac{\partial v}{\partial x})|_{x=0} =v_Y (y) (-\frac{1}{\delta_p} + \frac{(1+q)}{\delta_p}  ) $$

$$(\frac{\partial v}{\partial x})|_{x=0} = ( \frac{ v_Y (y) q}{\delta_p}  ) $$

substitute back:

$$\frac{d}{dy} \left[  \frac{v_Y^2 (y) \delta_p q^2  }{2(2+q)(1+ q)} \right] = - \nu  \frac{ v_Y (y) q}{\delta_p} + g\beta \int_0^X (T-T_\infty) dx$$

$$T-T_\infty = (T_s -T_\infty) \exp (-\frac{x}{\delta_T})$$

$$\int_0^X (T-T_\infty) = (T_s -T_\infty) \int_0^X dx \ \exp (-\frac{x}{\delta_T})$$

$$ = (T_s -T_\infty) (-\delta_T) \exp (-\frac{x}{\delta_T})|_{x=0}^{x=X}$$
$$ = (T_s -T_\infty) (-\delta_T) [\exp (-\frac{X}{\delta_T})-1]$$

$$ = (T_s -T_\infty) (\delta_T) [(1)]$$
$$ = (T_s -T_\infty) (\delta_T) $$

substitute back:

$$\frac{d}{dy} \left[  \frac{v_Y^2 (y) \delta_p q^2  }{2(2+q)(1+ q)} \right] = - \nu  \frac{ v_Y (y) q}{\delta_p} + g\beta  (T_s -T_\infty) (\delta_T)$$

\subsubsection{Momentum equation transformed... after substitution}
$$\frac{d}{dy} \left[  \frac{v_Y^2 (y) \delta_p q^2  }{2(2+q)(1+ q)} \right] = - \nu  \frac{ v_Y (y) q}{\delta_p} + g\beta  (T_s -T_\infty) \frac{\delta_p}{q}$$


\subsubsection{now onto the energy equation}

$$\frac{d}{dy} \int_0^X v (T_\infty - T) dx = \alpha (\frac{\partial T}{\partial x})|_{x=0}$$

$$-\frac{d}{dy} \int_0^X v (T- T_\infty ) dx = \alpha (\frac{\partial T}{\partial x})|_{x=0}$$


$$T-T_\infty = (T_s -T_\infty) \exp (-\frac{x}{\delta_T})$$
$$v (x,y) = v_Y (y) \exp (-\frac{x}{\delta_p}) (1- \exp (-\frac{x}{\delta_T}))$$

$$(T-T_\infty )v (x,y) = (T_s -T_\infty) \exp (-\frac{x}{\delta_T})v_Y (y) \exp (-\frac{x}{\delta_p}) (1- \exp (-\frac{x}{\delta_T}))$$

$$ = (T_s -T_\infty)v_Y (y) \exp (-\frac{x}{\delta_T}) \exp (-\frac{x}{\delta_p}) (1- \exp (-\frac{x}{\delta_T}))$$

$$ = (T_s -T_\infty)v_Y (y)  \exp (-\frac{x}{\delta_p}) (\exp (-\frac{x}{\delta_T})- \exp (-\frac{2x}{\delta_T}))$$

$$ = (T_s -T_\infty)v_Y (y)   (\exp (-x(\frac{1}{\delta_T}+\frac{1}{\delta_p}))- \exp (-x(\frac{2}{\delta_T}+\frac{1}{\delta_p})))$$

$$ = (T_s -T_\infty)v_Y (y)   (\exp (-x\frac{\delta_T + \delta_p}{\delta_T \delta_p})- \exp (-x\frac{\delta_T + 2\delta_p}{\delta_T \delta_p})$$

Now we can integrate


$$\int_0^X v (T_\infty - T) dx  = \int_0^X  dx \  (T_s -T_\infty)v_Y (y)   (\exp (-x\frac{\delta_T + \delta_p}{\delta_T \delta_p})- \exp (-x\frac{\delta_T + 2\delta_p}{\delta_T \delta_p})$$


$$\int_0^X v (T_\infty - T) dx  =(T_s -T_\infty)v_Y (y) \int_0^X  dx \     (\exp (-x\frac{\delta_T + \delta_p}{\delta_T \delta_p})- \exp (-x\frac{\delta_T + 2\delta_p}{\delta_T \delta_p})$$

$$\int_0^X v (T_\infty - T) dx  =(T_s -T_\infty)v_Y (y)   ( -\frac{\delta_T  \delta_p}{\delta_T + \delta_p} \exp (-x\frac{\delta_T + \delta_p}{\delta_T \delta_p})-(-\frac{\delta_T  \delta_p}{\delta_T + 2\delta_p}) \exp (-x\frac{\delta_T +2 \delta_p}{\delta_T \delta_p})|_{x=0}^{x=X}$$

$$\int_0^X v (T_\infty - T) dx  =(T_s -T_\infty)v_Y (y)   ( -\frac{\delta_T  \delta_p}{\delta_T + \delta_p} \exp (-x\frac{\delta_T + \delta_p}{\delta_T \delta_p})+(\frac{\delta_T  \delta_p}{\delta_T + 2\delta_p}) \exp (-x\frac{\delta_T +2 \delta_p}{\delta_T \delta_p})|_{x=0}^{x=X}$$


Evaluate at x=X, we find that it is negligible with respect to the first term.


$$\int_0^X v (T_\infty - T) dx  =(T_s -T_\infty)v_Y (y)   ( -\frac{\delta_T  \delta_p}{\delta_T + \delta_p} \exp (-x\frac{\delta_T + \delta_p}{\delta_T \delta_p})+(\frac{\delta_T  \delta_p}{\delta_T + 2\delta_p}) \exp (-x\frac{\delta_T +2 \delta_p}{\delta_T \delta_p})|_{x=0}^{x=X}$$

$$\int_0^X v (T_\infty - T) dx  =(T_s -T_\infty)v_Y (y) (-1)  ( -\frac{\delta_T  \delta_p}{\delta_T + \delta_p} \exp (-x\frac{\delta_T + \delta_p}{\delta_T \delta_p})+(\frac{\delta_T  \delta_p}{\delta_T + 2\delta_p}) \exp (-x\frac{\delta_T +2 \delta_p}{\delta_T \delta_p})|_{x=0}$$

$$\int_0^X v (T_\infty - T) dx  =(T_s -T_\infty)v_Y (y) (-1) ( -\frac{\delta_T  \delta_p}{\delta_T + \delta_p} +\frac{\delta_T  \delta_p}{\delta_T + 2\delta_p} )$$

We are going to reintroduce q, divide top and bottom by $\delta_T$



$$\int_0^X v (T_\infty - T) dx  =(T_s -T_\infty)v_Y (y) (-1) ( -\frac{ \delta_p}{1 + q} +\frac{  \delta_p}{1+2q} )$$

$$\int_0^X v (T_\infty - T) dx  =(T_s -T_\infty)v_Y (y) (-1) \delta_p ( -\frac{1 }{1 + q} +\frac{ 1}{1+2q} )$$

$$\int_0^X v (T_\infty - T) dx  =(T_s -T_\infty)v_Y (y) (-1) \delta_p ( \frac{-(1+2q)+(1+q) }{(1 + q)(1+2q)}  )$$


$$\int_0^X v (T_\infty - T) dx  =(T_s -T_\infty)v_Y (y) \delta_p (-1) ( \frac{-q}{(1 + q)(1+2q)}  )$$

substitute back


$$-\frac{d}{dy} \int_0^X v (T- T_\infty ) dx = \alpha (\frac{\partial T}{\partial x})|_{x=0}$$
$$-\frac{d}{dy} (-1) (T_s -T_\infty)v_Y (y) \delta_p ( \frac{-q}{(1 + q)(1+2q)}  ) = \alpha (\frac{\partial T}{\partial x})|_{x=0}$$

$$(T_s -T_\infty) \frac{d}{dy}  ( \frac{-q v_Y (y) \delta_p}{(1 + q)(1+2q)}  ) = \alpha (\frac{\partial T}{\partial x})|_{x=0}$$

divide both sides by $(T_s -T_\infty)$

$$(T_s -T_\infty) \frac{d}{dy}  ( \frac{q v_Y (y) \delta_p}{(1 + q)(1+2q)}  ) = -\alpha (\frac{\partial T}{\partial x})|_{x=0}$$

$$ \frac{d}{dy}  ( \frac{q v_Y (y) \delta_p}{(1 + q)(1+2q)}  ) = -\alpha (\frac{\partial \theta}{\partial x})|_{x=0}$$


$$T-T_\infty = (T_s -T_\infty) \exp (-\frac{x}{\delta_T})$$
$$\theta =  \exp (-\frac{x}{\delta_T})$$

$$(\frac{\partial \theta}{\partial x})|_{x=0} = [- \frac{1}{\delta_T} \exp (-\frac{x}{\delta_T})]_{x=0}=- \frac{1}{\delta_T} = - \frac{q}{\delta_p}$$

subs back to obtain:

$$ \frac{d}{dy}  ( \frac{q v_Y (y) \delta_p}{(1 + q)(1+2q)}  ) = \alpha \frac{q}{\delta_p}$$

One more assumption to make, q is independent of y since q = f(Pr) only.

$$ \frac{d}{dy}  ( \frac{ v_Y (y) \delta_p}{(1 + q)(1+2q)}  ) =  \frac{\alpha}{\delta_p}$$

\subsubsection{Energy and momentum equation for integral solution}


$$ \frac{d}{dy}  ( \frac{ v_Y (y) \delta_p}{(1 + q)(1+2q)}  ) =  \frac{\alpha}{\delta_p}$$


$$\frac{d}{dy} \left[  \frac{v_Y^2 (y) \delta_p q^2  }{2(2+q)(1+ q)} \right] = - \nu  \frac{ v_Y (y) q}{\delta_p} + g\beta  (T_s -T_\infty) \frac{\delta_p}{q}$$

Solution proceudre, integrate equations. we will need suitable BCs for $v_Y$, $\delta_p$ 

we don't really know how both of these will vary except for:

$$\delta_p \sim y^{1/4} \ ; v \sim y^{1/2}$$

$$\delta_p = A y^{1/4}$$

$$\int dt \delta_p = A y^{5/4} \frac{4}{5}= \delta_p \frac{4y}{5}+C$$

using this logic,

$$\int dy \frac{v_Y(y)}{\delta_p} =\int dy \frac{B y^{1/2}}{A y^{1/4}} =  C_2 + \frac{v_Y(y)}{\delta_p} \frac{4y}{5}$$


and also,

$$\int dy \frac{\alpha}{\delta_p} = \int dy \frac{\alpha}{A } = \frac{\alpha}{A} y^{3/4} \frac{4}{3} =  \frac{\alpha}{A y^{1/4}} y  \frac{4}{3} = \frac{\alpha}{\delta_p} \frac{4y}{3} $$

okay, so we will integrate from $y^*=0$ to $y^*=y$


$$ \frac{d}{dy}  ( \frac{ v_Y (y) \delta_p}{(1 + q)(1+2q)}  ) =  \frac{\alpha}{\delta_p}$$


$$\frac{d}{dy} \left[  \frac{v_Y^2 (y) \delta_p q^2  }{2(2+q)(1+ q)} \right] = - \nu  \frac{ v_Y (y) q}{\delta_p} + g\beta  (T_s -T_\infty) \frac{\delta_p}{q}$$

Integrating... with dummy variable $y^*$


$$ \frac{d}{dy}  ( \frac{ v_Y (y) \delta_p}{(1 + q)(1+2q)}  ) =  \frac{\alpha}{\delta_p}$$


$$\frac{d}{dy} \left[  \frac{v_Y^2 (y) \delta_p q^2  }{2(2+q)(1+ q)} \right] = - \nu  \frac{ v_Y (y) q}{\delta_p} + g\beta  (T_s -T_\infty) \frac{\delta_p}{q}$$


$$ \frac{d}{dy^*}  ( \frac{ v_Y (y^*) \delta_p}{(1 + q)(1+2q)}  ) =  \frac{\alpha}{\delta_p}$$


$$\frac{d}{dy^*} \left[  \frac{v_Y^2 (y^*) \delta_p q^2  }{2(2+q)(1+ q)} \right] = - \nu  \frac{ v_Y (y^*) q}{\delta_p} + g\beta  (T_s -T_\infty) \frac{\delta_p}{q}$$



$$ \int_0^y dy^* \  \frac{d}{dy^*}  ( \frac{ v_Y (y^*) \delta_p}{(1 + q)(1+2q)}  ) =  \int_0^y dy^* \ \frac{\alpha}{\delta_p}$$


$$\int_0^y dy^* \ \frac{d}{dy^*} \left[  \frac{v_Y^2 (y^*) \delta_p q^2  }{2(2+q)(1+ q)} \right] = -\int_0^y dy^* \  \nu  \frac{ v_Y (y^*) q}{\delta_p} + \int_0^y dy^* \  g\beta  (T_s -T_\infty) \frac{\delta_p}{q}$$

Some steps later... using the BC

at y=0
$$v_y(y)=0  \ \delta_p = 0$$

$$   ( \frac{ v_Y (y) \delta_p}{(1 + q)(1+2q)}  ) =  \int_0^y dy^* \ \frac{\alpha}{\delta_p}$$


$$\left[  \frac{v_Y^2 (y) \delta_p q^2  }{2(2+q)(1+ q)} \right] = -\int_0^y dy^* \  \nu  \frac{ v_Y (y^*) q}{\delta_p} + \int_0^y dy^* \  g\beta  (T_s -T_\infty) \frac{\delta_p}{q}$$

using what we derived,

$$   ( \frac{ v_Y (y) \delta_p}{(1 + q)(1+2q)}  ) = \frac{4y}{3} \frac{\alpha}{\delta_p}$$


$$\left[  \frac{v_Y^2 (y) \delta_p q^2  }{2(2+q)(1+ q)} \right] = -\frac{4y}{5} \nu  \frac{ v_Y (y) q}{\delta_p} + \frac{4y}{5}  g\beta  (T_s -T_\infty) \frac{\delta_p}{q}$$

Problem, we have 3 unknowns, q, $\delta_p$ and $v_y(y)$, we only have 2 equations. Where is equation 3?


%\section{solution procedure - similarity solution}
%
%Similarity variable
%
%$$\eta = \frac{x}{y} Ra_y^{1/4}$$
%
%(why? look at ND equation)
%
%We once again use the streamfunction
%
%$$u = \frac{\partial \psi}{\partial y}$$
%$$v = - \frac{\partial \psi}{\partial x}$$
%
%
%\section{Results}
%
%For high Pr fluids (\cite{bejan2013convection}) integral type solution, (for very large Pr)
%
%$$\frac{\delta}{\delta_T} = (\frac{6}{5}Pr)^{0.5}$$
%
%(local Nu)
%$$Nu = 0.783 Ra_y^{1/4}$$
%
%For similarity solution
%
%(local Nu)
%$$Nu = 0.503 Ra_y^{1/4}$$


\part{Resources Online}

For Momentum BLs:
\begin{verbatim}
http://web.mit.edu/fluids-modules/www/highspeed_flows/ver2/bl_Chap2.pdf
https://community.dur.ac.uk/suzanne.fielding/teaching/BLT/sec3.pdf

for Von Karman
https://nptel.ac.in/content/storage2/courses/112104118/lecture-29/29-3_momentum.htm
\end{verbatim}

For Thermal BLs:

\begin{verbatim}
http://mech.sut.ac.ir/People/Courses/18/Chapter3-%20Part2.pdf
http://raops.org.in/epapers/june15_9.pdf

https://nptel.ac.in/content/storage2/courses/112101001/downloads/lec25.pdf
\end{verbatim}

\part{Github Repo}
\begin{verbatim}
https://github.com/theodoreOnzGit/heatTransferTheory_YouTube
\end{verbatim}

Look under convection heat transfer...


\part{Bibliography}

\printbibliography

\end{document}

%% template for graphics
%
%\begin{figure}[H]
%\centering
%
%\includegraphics[width=\textwidth]{Q10_compare.png}
%\caption{H (blue) and Pb scattering inelastic (red) and elastic (green)}
%\label{H and Pb scattering}
%
%\end{figure}